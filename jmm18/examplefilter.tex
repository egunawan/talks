%!TEX root = main.tex
\def\myyscale{0.6}
\def\myxscale{1.1}
\def\id{id}
\frametitle{Why order filter (as opposed to order ideal)?}

Consider the % (indecomposable)
quiver representation $M$ over a field $k$

\vspace{-1mm}
\begin{tikzpicture}[yscale=\myyscale, xscale=\myxscale, %line join=bevel,
>=latex, 
font = \small
]
\def\posetedgecolor{red}
\node(1) at (0,0) {{$k$}}; %{$1$};
\node(2) at (1,-1) {{$k$}}; %{$2$};
\node(3) at (2,0) {{$k$}}; %{$3$};
\node(4) at (3,1) {{$k$}}; % {$4$};
\node(5) at (4,2) {{$k$}}; % {$5$};
\node(6) at (5,1) {{$k$}}; % {$6$};
\node(7) at (6,2) {{$k$}}; % {$7$};
\node(8) at (7,3) {{$k$}}; % {$8$};
\node(9) at (8,2) {{$k$}}; % {$9$};
\node(10) at (9,1) {{$k$}}; % {$10$};

%\node[\posetedgecolor] at (-0.9,0) {$\underline{1}$};
\draw[->] (2) -- (1) node[\posetedgecolor,pos=0.5,above] {$\id$}; 
\draw[->]  (2) -- (3) node[\posetedgecolor,pos=0.5,above] {$\id$}; 
\draw[->] (3) -- (4) node[\posetedgecolor,pos=0.5,above] {$\id$};  
\draw[->] (4) -- (5) node[\posetedgecolor,pos=0.5,above] {$\id$}; 
\draw[->] (6) -- (5) node[\posetedgecolor,pos=0.5,above] {$\id$}; 
\draw[->] (6) -- (7) node[\posetedgecolor,pos=0.5,above] {$\id$};  
\draw[->] (7) -- (8) node[\posetedgecolor,pos=0.5,above] {$\id$};  
\draw[->] (9) -- (8) node[\posetedgecolor,pos=0.5,above] {$\id$};  
\draw[->] (10) -- (9) node[\posetedgecolor,pos=0.5,above] {$\id$}; 
\end{tikzpicture}

\vspace{-1mm}
The subrepresentations of $M$ correspond to the order filters of the poset. \\[2mm]
Examples:   

\vspace{-5mm}
\begin{tikzpicture}[yscale=\myyscale, xscale=\myxscale, %line join=bevel,
>=latex, 
font = \small
]
\def\posetedgecolor{red}
\node(1) at (0,0) {{$0$}}; %{$1$};
\node(2) at (1,-1) {{$0$}}; %{$2$};
\node(3) at (2,0) {{$0$}}; %{$3$};
\node(4) at (3,1) {{$k$}}; % {$4$};
\node(5) at (4,2) {{$k$}}; % {$5$};
\node(6) at (5,1) {{$0$}}; % {$6$};
\node(7) at (6,2) {{$0$}}; % {$7$};
\node(8) at (7,3) {{$k$}}; % {$8$};
\node(9) at (8,2) {{$k$}}; % {$9$};
\node(10) at (9,1) {{$k$}}; % {$10$};

%\node[\posetedgecolor] at (-0.9,0) {$\underline{1}$};
\draw[->] (2) -- (1) node[\posetedgecolor,pos=0.5,above] {$0$}; 
\draw[->]  (2) -- (3) node[\posetedgecolor,pos=0.5,above] {$0$}; 
\draw[->] (3) -- (4) node[\posetedgecolor,pos=0.5,above] {$0$};  
\draw[->] (4) -- (5) node[\posetedgecolor,pos=0.5,above] {$\id$}; 
\draw[->] (6) -- (5) node[\posetedgecolor,pos=0.5,above] {$0$}; 
\draw[->] (6) -- (7) node[\posetedgecolor,pos=0.5,above] {$0$};  
\draw[->] (7) -- (8) node[\posetedgecolor,pos=0.5,above] {$0$};  
\draw[->] (9) -- (8) node[\posetedgecolor,pos=0.5,above] {$\id$};  
\draw[->] (10) -- (9) node[\posetedgecolor,pos=0.5,above] {$\id$}; 
\end{tikzpicture}

\vspace{-11mm}

\begin{tikzpicture}[yscale=\myyscale, xscale=\myxscale, %line join=bevel,
%>=latex
font = \small 
]
\def\posetedgecolor{red}
\node(1) at (0,0) {{$k$}}; %{$1$};
\node(2) at (1,-1) {{$0$}}; %{$2$};
\node(3) at (2,0) {{$k$}}; %{$3$};
\node(4) at (3,1) {{$k$}}; % {$4$};
\node(5) at (4,2) {{$k$}}; % {$5$};
\node(6) at (5,1) {{$0$}}; % {$6$};
\node(7) at (6,2) {{$k$}}; % {$7$};
\node(8) at (7,3) {{$k$}}; % {$8$};
\node(9) at (8,2) {{$k$}}; % {$9$};
\node(10) at (9,1) {{$0$}}; % {$10$};

%\node[\posetedgecolor] at (-0.9,0) {$\underline{1}$};
\draw[->] (2) -- (1) node[\posetedgecolor,pos=0.5,above] {$0$}; 
\draw[->] (2) -- (3) node[\posetedgecolor,pos=0.5,above] {$0$}; 
\draw[->] (3) -- (4) node[\posetedgecolor,pos=0.5,above] {$\id$};  
\draw[->] (4) -- (5) node[\posetedgecolor,pos=0.5,above] {$\id$}; 
\draw[->] (6) -- (5) node[\posetedgecolor,pos=0.5,above] {$0$}; 
\draw[->] (6) -- (7) node[\posetedgecolor,pos=0.5,above] {$0$};  
\draw[->] (7) -- (8) node[\posetedgecolor,pos=0.5,above] {$\id$};  
\draw[->] (9) -- (8) node[\posetedgecolor,pos=0.5,above] {$\id$};  
\draw[->] (10) -- (9) node[\posetedgecolor,pos=0.5,above] {$0$}; 
\end{tikzpicture}