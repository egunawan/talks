\documentclass[10pt,english,compress]{beamer}

\usepackage{tikz}
\usetikzlibrary{shapes,calc}
\usepackage{multicol}
\usepackage{ifthen}

\defbeamertemplate{footline}{author and frame number}{%
  \usebeamercolor[fg]{frame number in head/foot}%
  \usebeamerfont{frame number in head/foot}%
  \hspace{1em}\insertshortauthor\hfill
   {\insertshorttitle} -- 
   JMM 2018 
    \hfill  %
  %\insertframenumber/24
 % \insertframenumber\,/31\,\kern1em\vskip2pt%
\insertframenumber\,/\,\inserttotalframenumber
}
\setbeamertemplate{footline}[author and frame number]{}

\newcommand{\backupbegin}{
   \newcounter{framenumberappendix}
   \setcounter{framenumberappendix}{\value{framenumber}}
}
\newcommand{\backupend}{
   \addtocounter{framenumberappendix}{-\value{framenumber}}
   \addtocounter{framenumber}{\value{framenumberappendix}} 
}

\newcommand\myemph{\textbf}
\newtheorem{conjecture}[theorem]{Conjecture}
\newtheorem{proposition}[theorem]{Proposition}
\newtheorem{remark}[theorem]{Remark}
\newtheorem{furtherdirections}[theorem]{Further Directions}

\newenvironment<>{greenblock}[1]{%
  \begin{actionenv}#2%
      \def\insertblocktitle{#1}%
      \par%
      \mode<presentation>{%
        \setbeamercolor{block title}{fg=black,bg=green!20}
       \setbeamercolor{block body}{fg=black,bg=green!3}
      % \setbeamercolor{itemize item}{fg=orange!20!black}
       %\setbeamertemplate{itemize item}[triangle]
     }%
      \usebeamertemplate{block begin}}
    {\par\usebeamertemplate{block end}\end{actionenv}}


\newcommand*\circled[1]{\tikz[baseline=(char.base)]{
            \node[shape=circle,draw,inner sep=0.7pt] (char) {#1};}}
\newcommand*\doublecircled[1]{\tikz[baseline=(char.base)]{
            \node[shape=circle,draw,double,inner sep=0.9pt] (char) {#1};}}

\def\tfblank{$\underline{\phantom{xxxx}}$\quad}

\def\blank#1{\underline{\phantom{#1}}}

%My predefined color
\definecolor{myblue}{rgb}{0.1,0.15,0.7}

%Set color for Annotations
\colorlet{annotcol}{myblue!80!black}

%Store coordinates of points
\newcommand\tikzmark[1]{
  \tikz[remember picture,overlay] \coordinate (#1);
  }

%Command to add annotation above
\newcommand{\noteup}[3][0em,0em]{
\begin{tikzpicture}[
  remember picture,
  overlay]
\node[draw=annotcol,fill=white,ellipse,very thick,minimum width=2cm] 
  (mynode) 
  at ([shift=($({#1})+({0em,+5.5em})$)]{#2.north})
  {\begin{minipage}{2cm}\centering #3\end{minipage}};
\draw[annotcol,very thick,->,>=latex]
  (mynode.south) to[out=-90,in=+90] ([xshift=0.5em,yshift=1.3em]{#2}); 
\end{tikzpicture}
}

%Command to add annotation below
\newcommand{\notedown}[3][0em,0em]{
\begin{tikzpicture}[
  remember picture,
  overlay]
\node[draw=annotcol,fill=white,ellipse,very thick,minimum width=2cm] 
  (mynode) 
  at ([shift=($({#1})+({0em,-4em})$)]{#2.south})
  {\begin{minipage}{2cm}\centering #3\end{minipage}};
\draw[annotcol,very thick,->,>=latex]
  (mynode.north) to[out=90,in=-90] ([xshift=0.5em,yshift=-0.1em]{#2}); 
\end{tikzpicture}
}


%Beamer Slides
\begin{document}

\title[Cluster algebras and binary words]{Cluster algebras and binary words}  
\author[E. Gunawan]
{{ Emily Gunawan}\\
 {University of Connecticut}\\
 \texttt{egunawan.github.io} \\
} 
\date{
Joint Mathematics Meetings -- \\AMS Special Session on Dynamical Algebraic Combinatorics \\
Saturday, January 13, 2018
} 
\frame{\titlepage}

% Abstract: 
% Cluster algebras, introduced in 2000, are a class of commutative rings with rich combinatorial structure. They are connected to many areas of mathematics and physics, including representation theory, dynamical systems, geometry, and string theory.
%
%We establish a connection between cluster algebras from triangulated surfaces and binary words. Given a cluster variable, we associate to it a finite binary word. The subwords of this binary word are in bijection with the terms of the cluster variable. (Received September 25, 2017)

\frame{
\frametitle{Outline}
\begin{itemize}
\item Bijection between binary subwords and order filters of a poset
\item Bijection between binary subwords and perfect matchings of a snake graph  (terms of a cluster variable)
\end{itemize}
}

\frame{
\frametitle{Binary word}

\begin{definition}
\begin{itemize}
\item A \emph{binary word} is a finite sequence of letters belonging to $\{ 0, 1 \}$. 
{\bf In this talk, consider only words that start with $1$.}

\begin{itemize}
\item Example: \textcolor{blue}{10100}.
\end{itemize}

\item A \emph{subword} is a subsequence of a word.
\begin{itemize}
\item Example: Some subwords of \textcolor{blue}{10100} are \textcolor{blue}{the empty word}, \textcolor{blue}{11}, \textcolor{blue}{100}, and \textcolor{blue}{itself}.
\item Non-examples: 10001, 11000 are \emph{not} subwords of \textcolor{blue}{10100}.
\item Note: Even though \textcolor{blue}{1010} appears twice as a subsequence of \textcolor{blue}{10100}, we treat it as one subword. 

\end{itemize}
\end{itemize}
\end{definition}
}

\frame{
\frametitle{Subwords of 1011101100} 
\scriptsize 
\begin{multicols}{4}
\begin{enumerate}
\item \hfill empty
\item \hfill 1
\item \hfill 10
\item \hfill 11
\item \hfill 100
\item \hfill 101
\item \hfill 110
\item \hfill 111
\item \hfill 1000
\item \hfill 1001
\item \hfill 1010
\item \hfill 1011
\item \hfill 1100
\item \hfill 1101
\item \hfill 1110
\item \hfill 1111
\item \hfill 10000
\item \hfill 10010
\item \hfill 10011
\item \hfill 10100
\item \hfill 10101
\item \hfill 10110
\item \hfill 10111
\item \hfill 11000
\item \hfill 11010
\item \hfill 11011
\item \hfill 11100
\item \hfill 11101
\item \hfill 11110
\item \hfill 11111
\item \hfill 100100
\item \hfill 100110
\item \hfill 101000
\item \hfill 101010
\item \hfill 101011
\item \hfill 101100
\item \hfill 101101
\item \hfill 101110
\item \hfill 101111
\item \hfill 110100
\item \hfill 110110
\item \hfill 111000
\item \hfill 111010
\item \hfill 111011
\item \hfill 111100
\item \hfill 111101
\item \hfill 111110
\item \hfill 111111
\item \hfill 1001100
\item \hfill 1010100
\item \hfill 1010110
\item \hfill 1011000
\item \hfill 1011010
\item \hfill 1011011
\item \hfill 1011100
\item \hfill 1011101
\item \hfill 1011110
\item \hfill 1011111
\item \hfill 1101100
\item \hfill 1110100
\item \hfill 1110110
\item \hfill 1111000
\item \hfill 1111010
\item \hfill 1111011
\item \hfill 1111100
\item \hfill 1111110
\item \hfill 10101100
\item \hfill 10110100
\item \hfill 10110110
\item \hfill 10111000
\item \hfill 10111010
\item \hfill 10111011
\item \hfill 10111100
\item \hfill 10111110
\item \hfill 11101100
\item \hfill 11110100
\item \hfill 11110110
\item \hfill 11111100
\item \hfill 101101100
\item \hfill 101110100
\item \hfill 101110110
\item \hfill 101111100
\item \hfill 111101100
\item \hfill 1011101100
\end{enumerate}
\end{multicols}
}

\frame{
\frametitle{Binary word to poset}
Associate a subword $w=w_1 w_2 \dots w_n$ to the Hasse diagram of a ``line" poset with $n$ elements $V_1,V_2,\dots,V_n$
by assigning each $w_i$ $(i \geq 2)$ so that 

 $1$ corresponds to %the covering relation 
\begin{tikzpicture}[scale=0.5,>=latex]
\draw (-0.4,-0.2) node {$V_{i-1}$};
\draw (1.2,1.2) node {$V_i$};
\draw[ultra thick, ->] (0,0) to (1,1);
\end{tikzpicture}, and
$0$ corresponds to %the covering relation 
\begin{tikzpicture}[scale=0.5,>=latex]
\draw (-0.4,1.4) node {$V_{i-1}$};
\draw (1.2,-0.4) node {$V_i$};
\draw[ultra thick, <-] (0,1) to (1,0);
\end{tikzpicture}.\\[5mm]

Example: $w=\textcolor{blue}{1011101100}$

\begin{tikzpicture}[scale=1.1,%line join=bevel,
>=latex]
\def\posetedgecolor{blue}
\node(1) at (0,0) {\circled{$a$}}; %{$1$};
\node(2) at (1,-1) {\circled{$b$}}; %{$2$};
\node(3) at (2,0) {\circled{$c$}}; %{$3$};
\node(4) at (3,1) {\circled{$d$}}; % {$4$};
\node(5) at (4,2) {\circled{$e$}}; % {$5$};
\node(6) at (5,1) {\circled{$f$}}; % {$6$};
\node(7) at (6,2) {\circled{$g$}}; % {$7$};
\node(8) at (7,3) {\circled{$h$}}; % {$8$};
\node(9) at (8,2) {\circled{$i$}}; % {$9$};
\node(10) at (9,1) {\circled{$j$}}; % {$10$};

\node[\posetedgecolor] at (-0.9,0) {$\underline{1}$};
\draw[->] (2) -- (1) node[\posetedgecolor,pos=0.5,right,left] {$0$}; 
\draw[->]  (2) -- (3) node[\posetedgecolor,pos=0.5,right,left] {$1$}; ; 
\draw[->] (3) -- (4) node[\posetedgecolor,pos=0.5,right,left] {$1$}; ; 
\draw[->] (4) -- (5) node[\posetedgecolor,pos=0.5,right,left] {$1$}; ;
\draw[->] (6) -- (5) node[\posetedgecolor,pos=0.5,right,left] {$0$}; ; 
\draw[->] (6) -- (7) node[\posetedgecolor,pos=0.5,right,left] {$1$}; ; 
\draw[->] (7) -- (8) node[\posetedgecolor,pos=0.5,right,left] {$1$}; ; 
\draw[->] (9) -- (8) node[\posetedgecolor,pos=0.5,right,left] {$0$}; ; 
\draw[->] (10) -- (9)node[\posetedgecolor,pos=0.5,right,left] {$0$}; ;
\end{tikzpicture}
}


\frame{
\frametitle{Antichain}
\def\myxscale{0.99}
\def\myyscale{0.7}
An \emph{antichain} of a poset $P$ is a subset of $P$
such that no $2$ distinct elements are comparable. 
An antichain:

\vspace{-4mm}
\begin{tikzpicture}[yscale=\myyscale, xscale=\myxscale, %line join=bevel,
>=latex
]
\def\posetedgecolor{blue}
\node(1) at (0,0) {{$a$}}; %{$1$};
\node(2) at (1,-1) {{$b$}}; %{$2$};
\node(3) at (2,0) {{$c$}}; %{$3$};
\node(4) at (3,1) {\circled{$d$}}; % {$4$};
\node(5) at (4,2) {{$e$}}; % {$5$};
\node(6) at (5,1) {{$f$}}; % {$6$};
\node(7) at (6,2) {{$g$}}; % {$7$};
\node(8) at (7,3) {{$h$}}; % {$8$};
\node(9) at (8,2) {{$i$}}; % {$9$};
\node(10) at (9,1) {\circled{$j$}}; % {$10$};

\draw[->] (2) -- (1);
\draw[->] (2) -- (3);
\draw[->] (3) -- (4);
\draw[->] (4) -- (5);
\draw[->] (6) -- (5);
\draw[->] (6) -- (7);
\draw[->] (7) -- (8);
\draw[->] (9) -- (8);
\draw[->] (10) -- (9);
\end{tikzpicture}

% end an antichain

% begin not an antichain
\emph{Not} an antichain:
\vspace{-5mm}
\begin{tikzpicture}[yscale=\myyscale, xscale=\myxscale, %line join=bevel,
>=latex
]
\def\posetedgecolor{blue}
\node(1) at (0,0) {{$a$}}; %{$1$};
\node(2) at (1,-1) {{$b$}}; %{$2$};
\node(3) at (2,0) {\circled{$c$}}; %{$3$};
\node(4) at (3,1) {{$d$}}; % {$4$};
\node(5) at (4,2) {\circled{$e$}}; % {$5$};
\node(6) at (5,1) {{$f$}}; % {$6$};
\node(7) at (6,2) {{$g$}}; % {$7$};
\node(8) at (7,3) {{$h$}}; % {$8$};
\node(9) at (8,2) {{$i$}}; % {$9$};
\node(10) at (9,1) {\circled{$j$}}; % {$10$};

\draw[->] (2) -- (1); 
\draw[->]  (2) -- (3); 
\draw[->] (3) -- (4); 
\draw[->] (4) -- (5); 
\draw[->] (6) -- (5); 
\draw[->] (6) -- (7); 
\draw[->] (7) -- (8); 
\draw[->] (9) -- (8); 
\draw[->] (10) -- (9); 
\end{tikzpicture}
% end example of not an antichain %
}


\frame{
\frametitle{Bijection from antichains $A$ to subwords $s$ (G.)}
The empty antichain is mapped to the empty word.
Otherwise, map the antichain {\small $A=\{A_1$, $A_2$, $\dots$, $A_r\}$}  %, listed from left-to-right in order. 
to the following subword of $w$:
\begin{itemize}
\item $1$ is the first letter. \textcolor{blue}{$s=1\blank{01101100}$}
\item The next letters are the (possibly empty) sequence of edge labels between the first element of $P$ and $A_1$. $\textcolor{blue}{s=1\fbox{011}\blank{01100}}$
\end{itemize}
\begin{tikzpicture}[yscale=0.47, xscale=1.1, %line join=bevel,
>=latex,
%font = \footnotesize
font = \small
]
\def\posetedgecolor{blue}
\node(1) at (0,0) {{$a$}}; %{$1$};
\node(2) at (1,-1) {{$b$}}; %{$2$};
\node(3) at (2,0) {{$c$}}; %{$3$};
\node(4) at (3,1) {\circled{$d$}}; % {$4$};
\node(5) at (4,2) {{$e=M_1$}}; % {$5$};
\node(6) at (5,1) {{$f$}}; % {$6$};
\node(7) at (6,2) {{$g$}}; % {$7$};
\node(8) at (7,3) {{$h$}}; % {$8$};
\node(9) at (8,2) {{$i$}}; % {$9$};
\node(10) at (9,1) {\circled{$j$}}; % {$10$};

\node[\posetedgecolor] at (-0.3,0) {$\underline{1}$};
\draw[->] (2) -- (1) node[\posetedgecolor,pos=0.5] {$0$}; 
\draw[->]  (2) -- (3) node[\posetedgecolor,pos=0.5] {$1$}; 
\draw[->] (3) -- (4) node[\posetedgecolor,pos=0.5] {$1$};  
\draw[->] (4) -- (5); %node[\posetedgecolor,pos=0.5,right,left] {$1$}; 
\draw[->] (6) -- (5) node[\posetedgecolor,pos=0.5] {$0$}; 
\draw[->] (6) -- (7) node[\posetedgecolor,pos=0.5] {$1$};  
\draw[->] (7) -- (8) node[\posetedgecolor,pos=0.5] {$1$};  
\draw[->] (9) -- (8) node[\posetedgecolor,pos=0.5] {$0$};  
\draw[->] (10) -- (9) node[\posetedgecolor,pos=0.5] {$0$}; 
\end{tikzpicture}
\begin{itemize}
\item If $A$ contains only one element, we are done.
Otherwise, 
jump to the first element $M_1$ appearing after $A_1$ which is either minimal or maximal.   
The elements of $P$ between $A_1$ and $M_1$ (inclusive) are all comparable to $A_1$.
Since $A$ is an antichain, none of these are in $A$.

\item Record the labels of the edges between $M_1$ and $A_2$. $\textcolor{blue}{s=1011\boxed{01100}}$
\item Jump to the first element $M_2$ appearing after $A_2$ which is either minimal or maximal. Record the labels of the edges between $M_2$ and $A_3$, and so on.
\end{itemize}
}

\frame{
\frametitle{Bijection from antichains $A$ to subwords $s$ (G.)}
%Another example with $s=11010$ and 
Another example: 

The antichain $A=\{ A_1 = \circled{a}, A_2 = \circled{c},  A_3 = \circled{g}, A_4 = \circled{i} \}$ of $P$ 

is mapped to the subword 
\textcolor{blue}{$s=\underline{1} \, \framebox(7,12){} \, \framebox(10,12){1} \, \framebox(20,12){01} \, \framebox(10,12){0}$} of $w$.
\vspace{2mm}
\begin{tikzpicture}[
%yscale=0.5, xscale=0.99, %line join=bevel,
>=latex, 
%font = \small
]
\def\posetedgecolor{blue}
\node(1) at (0,0) {\circled{$a$}}; %{$1$};
\node(2) at (1,-1) {{$b=M_1$}}; %{$2$};
\node(3) at (2,0) {\circled{$c$}}; %{$3$};
\node(4) at (3,1) {{$d$}}; % {$4$};
\node(5) at (4,2) {{$e=M_2$}}; % {$5$};
\node(6) at (5,1) {{$f$}}; % {$6$};
\node(7) at (6,2) {\circled{$g$}}; % {$7$};
\node(8) at (7,3) {{$h=M_3$}}; % {$8$};
\node(9) at (8,2) {\circled{$i$}}; % {$9$};
\node(10) at (9,1) {{$j$}}; % {$10$};

\node[\posetedgecolor] at (-0.9,0) {$\underline{1}$};
\draw[->] (2) -- (1) ; %node[\posetedgecolor,pos=0.5] {$0$}; 
\draw[->]  (2) -- (3) node[\posetedgecolor,pos=0.5] {$1$}; 
\draw[->] (3) -- (4) ; %node[\posetedgecolor,pos=0.5] {$1$};  
\draw[->] (4) -- (5) ;%node[\posetedgecolor,pos=0.5,right,left] {$1$}; 
\draw[->] (6) -- (5) node[\posetedgecolor,pos=0.5] {$0$}; 
\draw[->] (6) -- (7) node[\posetedgecolor,pos=0.5] {$1$};  
\draw[->] (7) -- (8) ; %node[\posetedgecolor,pos=0.5] {$1$};  
\draw[->] (9) -- (8) node[\posetedgecolor,pos=0.5] {$0$};  
\draw[->] (10) -- (9); %node[\posetedgecolor,pos=0.5] {$0$}; 
\end{tikzpicture}
}



\frame{
\frametitle{Order filter}
\begin{definition}
An \emph{order filter} (or dual order ideal) is 
a subset $F$ of $P$ such that if $t \in F$ and $s \geq t$, then $s \in F$.
\end{definition}
\begin{itemize}
\item Fact: There is a one-to-one correspondence between antichains $A$ and order filters $F$, where $A$ is the set of minimal elements of $F$. \\[2mm]

\item Order filter examples: 
\end{itemize}
\vspace{-6mm}

\def\myyscale{0.6}
\def\myxscale{1.1}
\begin{tikzpicture}[yscale=\myyscale, xscale=\myxscale, %line join=bevel,
>=latex, 
font = \small
]
\def\posetedgecolor{blue}
\node(1) at (0,0) {{$a$}};
\node(2) at (1,-1) {{$b$}};
\node(3) at (2,0) {{$c$}};
\node(4) at (3,1) {\doublecircled{$d$}};
\node(5) at (4,2) {\circled{$e$}};
\node(6) at (5,1) {{$f$}};
\node(7) at (6,2) {{$g$}};
\node(8) at (7,3) {\circled{$h$}};
\node(9) at (8,2) {\circled{$i$}};
\node(10) at (9,1) {\doublecircled{$j$}}; 

\draw[->] (2) -- (1);
\draw[->]  (2) -- (3);
\draw[->] (3) -- (4);
\draw[->] (4) -- (5);
\draw[->] (6) -- (5);
\draw[->] (6) -- (7);
\draw[->] (7) -- (8);
\draw[->] (9) -- (8);
\draw[->] (10) -- (9);
\end{tikzpicture}

\vspace{-7mm}

\begin{tikzpicture}[yscale=\myyscale, xscale=\myxscale, %line join=bevel,
>=latex, 
font = \small 
]
\def\posetedgecolor{blue}
\node(1) at (0,0) {\doublecircled{$a$}};
\node(2) at (1,-1) {{$b$}};
\node(3) at (2,0) {\doublecircled{$c$}};
\node(4) at (3,1) {\circled{$d$}};
\node(5) at (4,2) {\circled{$e$}};
\node(6) at (5,1) {{$f$}};
\node(7) at (6,2) {\doublecircled{$g$}};
\node(8) at (7,3) {\circled{$h$}};
\node(9) at (8,2) {\doublecircled{$i$}};
\node(10) at (9,1) {{$j$}};

\draw[->] (2) -- (1);
\draw[->] (2) -- (3);
\draw[->] (3) -- (4);
\draw[->] (4) -- (5);
\draw[->] (6) -- (5);
\draw[->] (6) -- (7);
\draw[->] (7) -- (8);
\draw[->] (9) -- (8);
\draw[->] (10) -- (9);
\end{tikzpicture}

} % end frame


\frame{

%!TEX root = main.tex
\def\myyscale{0.6}
\def\myxscale{1.1}
\def\id{id}
\frametitle{Why order filter (as opposed to order ideal)?}

Consider the % (indecomposable)
quiver representation $M$ over a field $k$

\vspace{-1mm}
\begin{tikzpicture}[yscale=\myyscale, xscale=\myxscale, %line join=bevel,
>=latex, 
font = \small
]
\def\posetedgecolor{red}
\node(1) at (0,0) {{$k$}}; %{$1$};
\node(2) at (1,-1) {{$k$}}; %{$2$};
\node(3) at (2,0) {{$k$}}; %{$3$};
\node(4) at (3,1) {{$k$}}; % {$4$};
\node(5) at (4,2) {{$k$}}; % {$5$};
\node(6) at (5,1) {{$k$}}; % {$6$};
\node(7) at (6,2) {{$k$}}; % {$7$};
\node(8) at (7,3) {{$k$}}; % {$8$};
\node(9) at (8,2) {{$k$}}; % {$9$};
\node(10) at (9,1) {{$k$}}; % {$10$};

%\node[\posetedgecolor] at (-0.9,0) {$\underline{1}$};
\draw[->] (2) -- (1) node[\posetedgecolor,pos=0.5,above] {$\id$}; 
\draw[->]  (2) -- (3) node[\posetedgecolor,pos=0.5,above] {$\id$}; 
\draw[->] (3) -- (4) node[\posetedgecolor,pos=0.5,above] {$\id$};  
\draw[->] (4) -- (5) node[\posetedgecolor,pos=0.5,above] {$\id$}; 
\draw[->] (6) -- (5) node[\posetedgecolor,pos=0.5,above] {$\id$}; 
\draw[->] (6) -- (7) node[\posetedgecolor,pos=0.5,above] {$\id$};  
\draw[->] (7) -- (8) node[\posetedgecolor,pos=0.5,above] {$\id$};  
\draw[->] (9) -- (8) node[\posetedgecolor,pos=0.5,above] {$\id$};  
\draw[->] (10) -- (9) node[\posetedgecolor,pos=0.5,above] {$\id$}; 
\end{tikzpicture}

\vspace{-1mm}
The subrepresentations of $M$ correspond to the order filters of the poset. \\[2mm]
Examples:   

\vspace{-5mm}
\begin{tikzpicture}[yscale=\myyscale, xscale=\myxscale, %line join=bevel,
>=latex, 
font = \small
]
\def\posetedgecolor{red}
\node(1) at (0,0) {{$0$}}; %{$1$};
\node(2) at (1,-1) {{$0$}}; %{$2$};
\node(3) at (2,0) {{$0$}}; %{$3$};
\node(4) at (3,1) {{$k$}}; % {$4$};
\node(5) at (4,2) {{$k$}}; % {$5$};
\node(6) at (5,1) {{$0$}}; % {$6$};
\node(7) at (6,2) {{$0$}}; % {$7$};
\node(8) at (7,3) {{$k$}}; % {$8$};
\node(9) at (8,2) {{$k$}}; % {$9$};
\node(10) at (9,1) {{$k$}}; % {$10$};

%\node[\posetedgecolor] at (-0.9,0) {$\underline{1}$};
\draw[->] (2) -- (1) node[\posetedgecolor,pos=0.5,above] {$0$}; 
\draw[->]  (2) -- (3) node[\posetedgecolor,pos=0.5,above] {$0$}; 
\draw[->] (3) -- (4) node[\posetedgecolor,pos=0.5,above] {$0$};  
\draw[->] (4) -- (5) node[\posetedgecolor,pos=0.5,above] {$\id$}; 
\draw[->] (6) -- (5) node[\posetedgecolor,pos=0.5,above] {$0$}; 
\draw[->] (6) -- (7) node[\posetedgecolor,pos=0.5,above] {$0$};  
\draw[->] (7) -- (8) node[\posetedgecolor,pos=0.5,above] {$0$};  
\draw[->] (9) -- (8) node[\posetedgecolor,pos=0.5,above] {$\id$};  
\draw[->] (10) -- (9) node[\posetedgecolor,pos=0.5,above] {$\id$}; 
\end{tikzpicture}

\vspace{-11mm}

\begin{tikzpicture}[yscale=\myyscale, xscale=\myxscale, %line join=bevel,
%>=latex
font = \small 
]
\def\posetedgecolor{red}
\node(1) at (0,0) {{$k$}}; %{$1$};
\node(2) at (1,-1) {{$0$}}; %{$2$};
\node(3) at (2,0) {{$k$}}; %{$3$};
\node(4) at (3,1) {{$k$}}; % {$4$};
\node(5) at (4,2) {{$k$}}; % {$5$};
\node(6) at (5,1) {{$0$}}; % {$6$};
\node(7) at (6,2) {{$k$}}; % {$7$};
\node(8) at (7,3) {{$k$}}; % {$8$};
\node(9) at (8,2) {{$k$}}; % {$9$};
\node(10) at (9,1) {{$0$}}; % {$10$};

%\node[\posetedgecolor] at (-0.9,0) {$\underline{1}$};
\draw[->] (2) -- (1) node[\posetedgecolor,pos=0.5,above] {$0$}; 
\draw[->] (2) -- (3) node[\posetedgecolor,pos=0.5,above] {$0$}; 
\draw[->] (3) -- (4) node[\posetedgecolor,pos=0.5,above] {$\id$};  
\draw[->] (4) -- (5) node[\posetedgecolor,pos=0.5,above] {$\id$}; 
\draw[->] (6) -- (5) node[\posetedgecolor,pos=0.5,above] {$0$}; 
\draw[->] (6) -- (7) node[\posetedgecolor,pos=0.5,above] {$0$};  
\draw[->] (7) -- (8) node[\posetedgecolor,pos=0.5,above] {$\id$};  
\draw[->] (9) -- (8) node[\posetedgecolor,pos=0.5,above] {$\id$};  
\draw[->] (10) -- (9) node[\posetedgecolor,pos=0.5,above] {$0$}; 
\end{tikzpicture}

}


\frame{
\frametitle{Cluster algebras (Fomin and Zelevinsky, 2000)}
A \myemph{cluster algebra} is a subring of $\mathbb{Q}(x_1,\dots,x_n)$ with a distinguished set of generators, called \myemph{cluster variables}.\\

\begin{greenblock}{
Cluster algebras from surfaces %\\ 
{\footnotesize (Fomin, Shapiro, and Thurston, 2006, etc.)}
} 
\vspace{-2mm}
\begin{itemize}
\item 
A Riemann surface $S$ + %finitely many 
marked points gives rise to a cluster algebra.

\item Starting from a triangulation and initial cluster variables $x_1,\dots,x_n$,
produce all the other cluster variables by an iterative process called mutation.
%!TEX root = main.tex
\definecolor{mygreen}{cmyk}{0.5,0,0.5,0.5}
\begin{tikzpicture}[xscale = .6, yscale=0.3]
\tikzstyle{every node} = [font = \small]
\foreach \x in {0}
{
    \foreach \y in {-8}
    {
			\fill (\x,\y+1.75) circle (.05);
			%\fill (\x,\y+1.75) node [above] {\tiny{$1$}};
			\fill(\x+1.5158,\y+.875) circle (.05);
			%\fill (\x+1.5158,\y+.875) node [right] {\tiny{$2 $}};
			\fill (\x+1.5158,\y-.875) circle (.05);
			%\fill (\x+1.5158,\y-.875) node [right] {\tiny{$3$}};
			\fill(\x,\y-1.75) circle (.05);
			%\fill (\x,\y-1.75) node [below] {\tiny{$4$}};
			\fill(\x-1.5158,\y-.875) circle (.05);
			%\fill (\x-1.5158,\y-.875) node [left] {\tiny{$5$}};
			\fill (\x-1.5158,\y+.875) circle (.05);
			%\fill (\x-1.5158,\y+.875) node [left] {\tiny{$6$}};
		
			\draw [] (\x,\y+1.75)--(\x+1.5158,\y+.875); %v1 to v2
			\draw [] (\x+1.5158,\y+.875) -- (\x+1.5158,\y-.875); %v3 to v2
			\draw [] (\x+1.5158,\y-.875)--(\x,\y-1.75); %v4 to v3
			\draw [] (\x,\y-1.75) -- (\x-1.5158,\y-.875); %v5 to v4
			\draw [] (\x-1.5158,\y-.875)--(\x-1.5158,\y+.875); %v6 to v5
			\draw [] (\x-1.5158,\y+.875) -- (\x,\y+1.75); %v1 to v6
		
\draw [red, ultra thick] (\x,\y+1.75) -- (\x-1.5158,\y-.875) node[pos=0.5, left] {$x_1$};
\draw [blue, ultra thick] (\x,\y+1.75) -- (\x,\y-1.75) node[pos=0.5, left] {$x_2$};
\draw [mygreen, ultra thick] (\x,\y+1.75) -- (\x+1.5158,\y-.875) node[pos=0.7, left] {$x_3$};
	}
}
\end{tikzpicture}
\item
The cluster variables $\longleftrightarrow$ curves between marked points, called arcs. 
\end{itemize}
\end{greenblock}

\vspace{-2mm}

\myemph{Laurent Phenomenon} {\footnotesize (Fomin - Zelevinsky)} and \myemph{positivity} {\footnotesize (Lee - Schiffler, Gross - Hacking - Keel - Kontsevich, 2014, and special cases by others)}: Each cluster variable can be expressed as a Laurent polynomial in $\{x_1, \dots , x_n\}$, that is, as
\vspace{-2mm}
\[
\frac{f(x_1,\dots,x_n)}{x_1^{d_1} \dots x_n^{d_n}},
\]
where $f$ is a polynomial with positive coefficients.
}

\frame{
\frametitle{Snake graphs}
\begin{definition}
A \emph{snake graph} is a connected sequence of square tiles
\begin{tikzpicture}[scale=.5]
\draw (0,0) -- (1,0) -- (1,1) -- (0,1) -- cycle;
\end{tikzpicture}.
To build a snake graph, start with one tile, then glue a new tile to the north or the east of the previous tile.
\end{definition}
 
\vspace{2mm}
Example of a snake graph with $10$ tiles: 
 
 % begin snake graph 1011101100 or 2,3,1,2,3
 \begin{tikzpicture}[scale=0.7]
 \foreach \x in {0,1,...,2} % bottom row 
 {
     \draw (\x + 0, 0) -- (\x+1,0) -- (\x+1,1)-- (\x+0,1) -- cycle;
 }

\def\y{1}
 \foreach \x in {2,...,5} % second from bottom row
 {
     \draw (\x + 0, \y) -- (\x+1,\y) -- (\x+1,\y+1)-- (\x+0,\y+1) -- cycle;
 } 
 
\def\y{2}
 \foreach \x in {5} % third from bottom row
 {
     \draw (\x + 0, \y) -- (\x+1,\y) -- (\x+1,\y+1)-- (\x+0,\y+1) -- cycle;
 }  

\def\y{3}
\foreach \x in {5,6} % fourth from bottom row
{
     \draw (\x + 0, \y) -- (\x+1,\y) -- (\x+1,\y+1)-- (\x+0,\y+1) -- cycle;
 } 
 
\end{tikzpicture}

\vspace{2mm}
\begin{itemize}
\item History: Used by Musiker, Propp, Schiffler, and Williams to study positivity and bases of cluster algebras from surfaces (2005, 2009--10). The theory of abstract snake graph was developed further by {\c{C}anak\c{c}\i}, Lee, and Schiffler (2012--17). 
\end{itemize}
} % end frame


\frame{
\frametitle{Sign function}
\begin{definition}[{\c{C}anak\c{c}\i}, Schiffler]
A \emph{sign function} on a snake graph $G$ is a map from the set of edges of G to 
$\{+,-\}$ 
such that, for every tile of $G$, 
\begin{itemize}
\item the north and the west edges have the same sign, 
\item the south and the east edges have the same sign, and 
%begins just two tiles, one with positive epsilon and one with negative epsilon
\vspace{-2mm}
\begin{center} \begin{tikzpicture}[scale=0.45,font=\scriptsize]
\def\y{0}
\def\south{$-$}
\def\east{$-$}
\def\north{$+$}
\def\west{$+$}
 \foreach \x in {0}
 {
     \draw (\x + 0, \y) --  (\x+1,\y) node [pos=0.5,below] {\south} -- (\x+1,\y+1) node [pos=0.5,right] {\east} -- (\x+0,\y+1) node [pos=0.5,above] {\north} --  (\x + 0, \y)  node [pos=0.5,left] {\west};
 }  
 
\def\south{$+$}
\def\east{$+$}
\def\north{$-$}
\def\west{$-$}
 \foreach \x in {5}
 {
     \draw (\x + 0, \y) --  (\x+1,\y) node [pos=0.5,below] {\south} -- (\x+1,\y+1) node [pos=0.5,right] {\east} -- (\x+0,\y+1) node [pos=0.5,above] {\north} --  (\x + 0, \y)  node [pos=0.5,left] {\west};
 }   
\end{tikzpicture} \end{center}
\vspace{-4mm}
\item the sign on the north edge is opposite to the sign on the south edge.
\end{itemize} 
\vspace{-1mm}
\newcommand\macropositivetile{
     \def\south{$-$}
     \def\east{$-$}
     \def\north{$+$}
     \def\west{$+$} 
     \draw (\x + 0, \y) --  (\x+1,\y) node [pos=0.5,above] {\south} -- (\x+1,\y+1) node [pos=0.5,left] {\east} -- (\x+0,\y+1) node [pos=0.5,above] {\north} --  (\x + 0, \y)  node [pos=0.5,left] {\west};
}
\newcommand\macropositivetileBinary{
     \def\south{$0$}
     \def\east{$0$}
     \def\north{$1$}
     \def\west{$1$} 
     \draw (\x + 0, \y) --  (\x+1,\y) node [pos=0.5,above] {\south} -- (\x+1,\y+1) node [pos=0.5,left] {\east} -- (\x+0,\y+1) node [pos=0.5,above] {\north} --  (\x + 0, \y)  node [pos=0.5,left] {\west};
}
\newcommand\macronegativetile{
     \def\south{$+$}
     \def\east{$+$}
     \def\north{$-$}
     \def\west{$-$} 
     \draw (\x + 0, \y) --  (\x+1,\y) node [pos=0.5,above] {\south} -- (\x+1,\y+1) node [pos=0.5,left] {\east} -- (\x+0,\y+1) node [pos=0.5,above] {\north} --  (\x + 0, \y)  node [pos=0.5,left] {\west};
}
\newcommand\macronegativetileBinary{
     \def\south{$1$}
     \def\east{$1$}
     \def\north{$0$}
     \def\west{$0$} 
     \draw (\x + 0, \y) --  (\x+1,\y) node [pos=0.5,above] {\south} -- (\x+1,\y+1) node [pos=0.5,left] {\east} -- (\x+0,\y+1) node [pos=0.5,above] {\north} --  (\x + 0, \y)  node [pos=0.5,left] {\west};
}

% begin snake graph 1011101100 or 2,3,1,2,3
 \begin{tikzpicture}[scale=0.65,font=\tiny]
 \foreach \x in {0,1,...,2} % bottom row 
 {
%     \draw (\x + 0, 0) -- (\x+1,0) -- (\x+1,1)-- (\x+0,1) -- cycle;
\def\y{0}
     \ifthenelse{\x = 0 \OR \x =2}
     {
     \macropositivetile
     }{
     \macronegativetile
     }
 }

\def\y{1}
 \foreach \x in {2,...,5} % second from bottom row
 {
     %\draw (\x + 0, \y) -- (\x+1,\y) -- (\x+1,\y+1)-- (\x+0,\y+1) -- cycle;
     \ifthenelse{\x = 3 \OR \x =5}
     {\macropositivetile}{\macronegativetile}

 } 
 
\def\y{2}
 \foreach \x in {5} % third from bottom row
 {
     %\draw (\x + 0, \y) -- (\x+1,\y) -- (\x+1,\y+1)-- (\x+0,\y+1) -- cycle;
     \macronegativetile
 }  

\def\y{3}
\foreach \x in {5,6} % fourth from bottom row
{
     %\draw (\x + 0, \y) -- (\x+1,\y) -- (\x+1,\y+1)-- (\x+0,\y+1) -- cycle;
     \ifthenelse{\x = 5}
     {\macropositivetile}{\macronegativetile}
 } 
 
\end{tikzpicture}
% begin snake graph 1011101100 or 2,3,1,2,3 (with 1 0 signs)
 \begin{tikzpicture}[scale=0.65,font=\tiny]
 
%       \ifthenelse{2 > 1}% 
%         {\draw[ultra thick,blue,->] (0,0) -- (1*1,1*1);
%          \foreach \x in {.0,.1,...,5} \draw[blue] 
%                                    (1*\x,1*\x-\x*\x) circle (3pt);}{}
 
 \foreach \x in {0,1,...,2} % bottom row 
 {
%     \draw (\x + 0, 0) -- (\x+1,0) -- (\x+1,1)-- (\x+0,1) -- cycle;
\def\y{0}
     \ifthenelse{\x = 0 \OR \x =2}
     {
     \macropositivetileBinary
     }{
     \macronegativetileBinary
     }
 }

\def\y{1}
 \foreach \x in {2,...,5} % second from bottom row
 {
     %\draw (\x + 0, \y) -- (\x+1,\y) -- (\x+1,\y+1)-- (\x+0,\y+1) -- cycle;
     \ifthenelse{\x = 3 \OR \x =5}
     {\macropositivetileBinary}{\macronegativetileBinary}

 } 
 
\def\y{2}
 \foreach \x in {5} % third from bottom row
 {
     %\draw (\x + 0, \y) -- (\x+1,\y) -- (\x+1,\y+1)-- (\x+0,\y+1) -- cycle;
     \macronegativetileBinary
 }  

\def\y{3}
\foreach \x in {5,6} % fourth from bottom row
{
     %\draw (\x + 0, \y) -- (\x+1,\y) -- (\x+1,\y+1)-- (\x+0,\y+1) -- cycle;
     \ifthenelse{\x = 5}
     {\macropositivetileBinary}{\macronegativetileBinary}
 } 
 
\end{tikzpicture}
Note: There are two possible sign functions on $G$.

For convenience, we replace $+$ with $1$ and $-$ with $0$.
\end{definition}

} % end frame



\frame{
\frametitle{Sign Sequence}
\def\signsequencescale{0.79}
The sequence of signs of the interior edges corresponds to a binary word.

\newcommand\macropositivetileBinaryFirstTileWestEast{
     \def\south{}
     \def\east{\textcolor{blue}{$0$}}
     \def\north{}
     \def\west{\underline{$1$}}
     \draw (\x + 0, \y) --  (\x+1,\y) node [pos=0.5,above] {\south} -- (\x+1,\y+1) node [pos=0.5,left] {\east} -- (\x+0,\y+1) node [pos=0.5,above] {\north} --  (\x + 0, \y)  node [pos=0.5,left] {\west};
}
\newcommand\macropositivetileBinaryWestEast{
     \def\south{}
     \def\east{\textcolor{blue}{$0$}}
     \def\north{}
     \def\west{\textcolor{blue}{$1$}} 
     \draw (\x + 0, \y) --  (\x+1,\y) node [pos=0.5,above] {\south} -- (\x+1,\y+1) node [pos=0.5,left] {\east} -- (\x+0,\y+1) node [pos=0.5,above] {\north} --  (\x + 0, \y)  node [pos=0.5,left] {\west};
}
\newcommand\macropositivetileBinaryWestNorth{
     \def\south{}
     \def\east{}
     \def\north{\textcolor{blue}{$1$}}
     \def\west{\textcolor{blue}{$1$}} 
     \draw (\x + 0, \y) --  (\x+1,\y) node [pos=0.5,above] {\south} -- (\x+1,\y+1) node [pos=0.5,left] {\east} -- (\x+0,\y+1) node [pos=0.5,above] {\north} --  (\x + 0, \y)  node [pos=0.5,left] {\west};
}
\newcommand\macropositivetileBinarySouthEast{
     \def\south{\textcolor{blue}{$0$}}
     \def\east{\textcolor{blue}{$0$}}
     \def\north{}
     \def\west{} 
     \draw (\x + 0, \y) --  (\x+1,\y) node [pos=0.5,above] {\south} -- (\x+1,\y+1) node [pos=0.5,left] {\east} -- (\x+0,\y+1) node [pos=0.5,above] {\north} --  (\x + 0, \y)  node [pos=0.5,left] {\west};
}
\newcommand\macropositivetileBinarySouthNorth{
     \def\south{\textcolor{blue}{$0$}}
     \def\east{}
     \def\north{\textcolor{blue}{$1$}}
     \def\west{} 
     \draw (\x + 0, \y) --  (\x+1,\y) node [pos=0.5,above] {\south} -- (\x+1,\y+1) node [pos=0.5,left] {\east} -- (\x+0,\y+1) node [pos=0.5,above] {\north} --  (\x + 0, \y)  node [pos=0.5,left] {\west};
}
\newcommand\macronegativetileBinary{
     \def\south{$1$}
     \def\east{$1$}
     \def\north{$0$}
     \def\west{$0$} 
     \draw (\x + 0, \y) --  (\x+1,\y) node [pos=0.5,above] {\south} -- (\x+1,\y+1) node [pos=0.5,left] {\east} -- (\x+0,\y+1) node [pos=0.5,above] {\north} --  (\x + 0, \y)  node [pos=0.5,left] {\west};
}
\newcommand\macronegativetileBinaryWestEast{
     \def\south{}
     \def\east{\textcolor{blue}{$1$}}
     \def\north{}
     \def\west{\textcolor{blue}{$0$}}
     \draw (\x + 0, \y) --  (\x+1,\y) node [pos=0.5,above] {\south} -- (\x+1,\y+1) node [pos=0.5,left] {\east} -- (\x+0,\y+1) node [pos=0.5,above] {\north} --  (\x + 0, \y)  node [pos=0.5,left] {\west};
}
\newcommand\macronegativetileBinaryWestNorth{
     \def\south{}
     \def\east{}
     \def\north{\textcolor{blue}{$0$}}
     \def\west{\textcolor{blue}{$0$}} 
     \draw (\x + 0, \y) --  (\x+1,\y) node [pos=0.5,above] {\south} -- (\x+1,\y+1) node [pos=0.5,left] {\east} -- (\x+0,\y+1) node [pos=0.5,above] {\north} --  (\x + 0, \y)  node [pos=0.5,left] {\west};
}
\newcommand\macronegativetileBinarySouthEast{
     \def\south{\textcolor{blue}{$1$}}
     \def\east{\textcolor{blue}{$1$}}
     \def\north{}
     \def\west{} 
     \draw (\x + 0, \y) --  (\x+1,\y) node [pos=0.5,above] {\south} -- (\x+1,\y+1) node [pos=0.5,left] {\east} -- (\x+0,\y+1) node [pos=0.5,above] {\north} --  (\x + 0, \y)  node [pos=0.5,left] {\west};
}
\newcommand\macronegativetileBinarySouthNorth{
     \def\south{\textcolor{blue}{$1$}}
     \def\east{}
     \def\north{\textcolor{blue}{$0$}}
     \def\west{} 
     \draw (\x + 0, \y) --  (\x+1,\y) node [pos=0.5,above] {\south} -- (\x+1,\y+1) node [pos=0.5,left] {\east} -- (\x+0,\y+1) node [pos=0.5,above] {\north} --  (\x + 0, \y)  node [pos=0.5,left] {\west};
}
\newcommand\macronegativetileBinaryWest{ % for a last tile
     \def\south{}
     \def\east{}
     \def\north{}
     \def\west{\textcolor{blue}{$0$}} 
     \draw (\x + 0, \y) --  (\x+1,\y) node [pos=0.5,above] {\south} -- (\x+1,\y+1) node [pos=0.5,left] {\east} -- (\x+0,\y+1) node [pos=0.5,above] {\north} --  (\x + 0, \y)  node [pos=0.5,left] {\west};
}

% begin snake graph 1011101100 or 2,3,1,2,3 (with 1 0 signs in interior)
 \begin{tikzpicture}[scale=\signsequencescale,font=\small]
 \foreach \x in {0,1,...,2} % bottom row 
 {
\def\y{0}
     \ifthenelse{\x = 0}{\macropositivetileBinaryFirstTileWestEast}{}
     \ifthenelse{\x = 1}{\macronegativetileBinaryWestEast}{}
     \ifthenelse{\x =2}{\macropositivetileBinaryWestNorth}{}
 }

\def\y{1}
 \foreach \x in {2,...,5} % second from bottom row
 {
     \ifthenelse{\x = 2}
     {\macronegativetileBinarySouthEast}{}
     \ifthenelse{\x = 3}
     {\macropositivetileBinaryWestEast}{}
     \ifthenelse{\x = 4}
     {\macronegativetileBinaryWestEast}{}
     \ifthenelse{\x = 5}
     {\macropositivetileBinaryWestNorth}{}
 } 
 
\def\y{2}
 \foreach \x in {5} % third from bottom row
 {
     \macronegativetileBinarySouthNorth
 }  

\def\y{3}
\foreach \x in {5,6} % fourth from bottom row
{
     \ifthenelse{\x = 5}{\macropositivetileBinarySouthEast}{}
     \ifthenelse{\x = 6}{\macronegativetileBinaryWest}{}
 } 
 
\end{tikzpicture}
Example: the word $\underline{1}\textcolor{blue}{011101100}$.
} % end frame 


\frame{
\frametitle{Perfect matchings}
\begin{definition}
A \emph{matching} of a graph $G$ is a subset of non-adjacent edges of the graph.
A \emph{perfect matching} of $G$ is a matching where every vertex of $G$ is adjacent to exactly one edge of the matching. 

\def\myscale{0.55}
\newcommand\macropositivetile{
     \def\south{}
     \def\east{}
     \def\north{}
     \def\west{} 
     \draw (\x + 0, \y) --  (\x+1,\y) node [pos=0.5,above] {\south} -- (\x+1,\y+1) node [pos=0.5,left] {\east} -- (\x+0,\y+1) node [pos=0.5,above] {\north} --  (\x + 0, \y)  node [pos=0.5,left] {\west};
}

\newcommand\macronegativetile{
     \def\south{}
     \def\east{}
     \def\north{}
     \def\west{} 
     \draw (\x + 0, \y) --  (\x+1,\y) node [pos=0.5,above] {\south} -- (\x+1,\y+1) node [pos=0.5,left] {\east} -- (\x+0,\y+1) node [pos=0.5,above] {\north} --  (\x + 0, \y)  node [pos=0.5,left] {\west};
}

\newcommand\matchingcolor{3pt}

% begin snake graph 1011101100 or 2,3,1,2,3 with minimal matching
 \begin{tikzpicture}[scale=\myscale,font=\tiny]
 \foreach \x in {0,1,...,2} % bottom row 
 {
%     \draw (\x + 0, 0) -- (\x+1,0) -- (\x+1,1)-- (\x+0,1) -- cycle;
\def\y{0}
     \ifthenelse{\x = 0 \OR \x =2}
     {
     \macropositivetile
     }{
     \macronegativetile
     }
     \ifthenelse{\x = 0 \OR \x =2}{\draw[line width=\matchingcolor] (\x + 0, \y+0) -- (\x+1,\y+0);}{} % south
     \ifthenelse{\x = 0}{\draw[line width=\matchingcolor] (\x + \y+0, \y+1) -- (\x+1,\y+1);}{} % north
 }

\def\y{1}
 \foreach \x in {2,...,5} % second from bottom row
 {
     %\draw (\x + 0, \y) -- (\x+1,\y) -- (\x+1,\y+1)-- (\x+0,\y+1) -- cycle;
     \ifthenelse{\x = 3 \OR \x =5}
     {\macropositivetile}{\macronegativetile}
     \ifthenelse{\x = 2}{\draw[line width=\matchingcolor] (\x+0, \y+0) -- (\x, \y+1);}{} % west
     \ifthenelse{\x = 3 \OR \x =5}{\draw[line width=\matchingcolor] (\x+0, \y+0) -- (\x+1, \y+0);}{} % south
     \ifthenelse{\x = 3}{\draw[line width=\matchingcolor] (\x+0, \y+1) -- (\x+1,\y+1);}{} % north
 } 
 
\def\y{2}
 \foreach \x in {5} % third from bottom row
 {
     %\draw (\x + 0, \y) -- (\x+1,\y) -- (\x+1,\y+1)-- (\x+0,\y+1) -- cycle;
     \macronegativetile
     \ifthenelse{\x = 5}{\draw[line width=\matchingcolor] (\x+0, \y+0) -- (\x, \y+1);}{} % west
     \ifthenelse{\x = 5}{\draw[line width=\matchingcolor] (\x+1, \y) -- (\x+1,\y+1);}{} % east
 }  

\def\y{3}
\foreach \x in {5,6} % fourth from bottom row
{
     %\draw (\x + 0, \y) -- (\x+1,\y) -- (\x+1,\y+1)-- (\x+0,\y+1) -- cycle;
     \ifthenelse{\x = 5}
     {\macropositivetile}{\macronegativetile}
     \ifthenelse{\x = 5}{\draw[line width=\matchingcolor] (\x+0, \y+1) -- (\x+1,\y+1);}{} % north
     \ifthenelse{\x = 6}{\draw[line width=\matchingcolor] (\x+1, \y) -- (\x+1,\y+1);}{} % east
 } 
\end{tikzpicture} % end of minimal matching
% begin snake graph 1011101100 or 2,3,1,2,3 with matching generated by d and k
 \begin{tikzpicture}[scale=\myscale,font=\tiny]
 \foreach \x in {0,1,...,2} % bottom row 
 {
%     \draw (\x + 0, 0) -- (\x+1,0) -- (\x+1,1)-- (\x+0,1) -- cycle;
\def\y{0}
     \ifthenelse{\x = 0 \OR \x =2}
     {
     \macropositivetile
     }{
     \macronegativetile
     }
     \ifthenelse{\x = 0 \OR \x =2}{\draw[line width=\matchingcolor] (\x + 0, \y+0) -- (\x+1,\y+0);}{} % south
     \ifthenelse{\x = 0}{\draw[line width=\matchingcolor] (\x + \y+0, \y+1) -- (\x+1,\y+1);}{} % north
 }

\def\y{1}
 \foreach \x in {2,...,5} % second from bottom row
 {
     %\draw (\x + 0, \y) -- (\x+1,\y) -- (\x+1,\y+1)-- (\x+0,\y+1) -- cycle;
     \ifthenelse{\x = 3 \OR \x =5}
     {\macropositivetile}{\macronegativetile}
%     \ifthenelse{\x = 2}{\draw[line width=\matchingcolor] (\x+0, \y+0) -- (\x, \y+1);}{} % west
     \ifthenelse{\x = 2 \OR \x =5}{\draw[line width=\matchingcolor] (\x+0, \y+0) -- (\x+1, \y+0);}{} % south
     \ifthenelse{\x = 2}{\draw[line width=\matchingcolor] (\x+0, \y+1) -- (\x+1,\y+1);}{} % north
      \ifthenelse{\x = 3}{\draw[line width=\matchingcolor] (\x+1, \y) -- (\x+1,\y+1);}{} % east
 } 
 
\def\y{2}
 \foreach \x in {5} % third from bottom row
 {
     %\draw (\x + 0, \y) -- (\x+1,\y) -- (\x+1,\y+1)-- (\x+0,\y+1) -- cycle;
     \macronegativetile
%     \ifthenelse{\x = 5}{\draw[line width=\matchingcolor] (\x+0, \y+0) -- (\x, \y+1);}{} % west
%     \ifthenelse{\x = 5}{\draw[line width=\matchingcolor] (\x+1, \y) -- (\x+1,\y+1);}{} % east
     \ifthenelse{\x = 5}{\draw[line width=\matchingcolor] (\x+0, \y+0) -- (\x+1, \y+0);}{} % south
 }  

\def\y{3}
\foreach \x in {5,6} % fourth from bottom row
{
     %\draw (\x + 0, \y) -- (\x+1,\y) -- (\x+1,\y+1)-- (\x+0,\y+1) -- cycle;
     \ifthenelse{\x = 5}
     {\macropositivetile}{\macronegativetile}
%     \ifthenelse{\x = 5}{\draw[line width=\matchingcolor] (\x+0, \y+1) -- (\x+1,\y+1);}{} % north
        \ifthenelse{\x = 5}{\draw[line width=\matchingcolor] (\x+0, \y+0) -- (\x, \y+1);}{} % west

%     \ifthenelse{\x = 6}{\draw[line width=\matchingcolor] (\x+1, \y) -- (\x+1,\y+1);}{} % east
     \ifthenelse{\x = 6}{\draw[line width=\matchingcolor] (\x+0, \y+0) -- (\x+1, \y+0);}{} % south
     \ifthenelse{\x = 6}{\draw[line width=\matchingcolor] (\x+0, \y+1) -- (\x+1,\y+1);}{} % north
 } 
\end{tikzpicture} % end of matching subword with 1 missing
\end{definition}

\begin{theorem}[{\small Musiker, Schiffler, and Williams, 2009--10}]
Each cluster variable (of a cluster algebra from a surface) can be written as the sum of the weights of all perfect matchings of a certain snake graph. 
\\
(Note: this demonstrates that the Laurent polynomial expansion has all positive coefficients).
\end{theorem}
} % end frame


\frame{
\frametitle{Bijection from subwords to perfect matchings (G.)}
\begin{itemize}
\item 
Highlight the internal edges of $G$ corresponding to $s$.

Example: $w = 1011101100$ and subword $\textcolor{blue}{s = 101101100}$.

Highlight the edges $\underline{1}\, \underline{0} \, \underline{1} \, \underline{1} \, 1 \, \underline{0} \, \underline{1} \, \underline{1} \, \underline{0} \, \underline{0}$ below.

\item For each path $L$ of consecutive highligted edges, let $\square_L$ be the tile which is north/east of the last edge in $L$. 
%Example: subword $s = 101101100$.
\end{itemize}

\newcommand\macrotile[1]{
     \def\south{}
     \def\east{}
     \def\north{}
     \def\west{} 
     \draw (\x + 0, \y) --  (\x+1,\y) node [pos=0.5,above] {\south} -- (\x+1,\y+1) node [pos=0.5,left] {\east} -- (\x+0,\y+1) node [pos=0.5,above] {\north} --  (\x + 0, \y)  node [pos=0.5,left] {\west};
}
\newcommand\macropositivetileBinary{
     \def\south{$0$}
     \def\east{$0$}
     \def\north{$1$}
     \def\west{$1$} 
     \draw (\x + 0, \y) --  (\x+1,\y) node [pos=0.5,above] {\south} -- (\x+1,\y+1) node [pos=0.5,left] {\east} -- (\x+0,\y+1) node [pos=0.5,above] {\north} --  (\x + 0, \y)  node [pos=0.5,left] {\west};
}
\newcommand\macropositivetileBinaryWestEast{
     \def\south{}
     \def\east{\textcolor{blue}{\circled{$0$}}}
     \def\north{}
     \def\west{\textcolor{blue}{\circled{$1$}}}
     \draw (\x + 0, \y) --  (\x+1,\y) node [pos=0.5] {\south} -- (\x+1,\y+1) node [pos=0.5] {\east} -- (\x+0,\y+1) node [pos=0.5] {\north} --  (\x + 0, \y)  node [pos=0.5] {\west};
}
\newcommand\macropositivetileBinaryEast{
     \def\south{}
     \def\east{\textcolor{blue}{\circled{$0$}}}
     \def\north{}
     \def\west{{$1$}} 
     \draw (\x + 0, \y) --  (\x+1,\y) node [pos=0.5] {\south} -- (\x+1,\y+1) node [pos=0.5] {\east} -- (\x+0,\y+1) node [pos=0.5] {\north} --  (\x + 0, \y)  node [pos=0.5] {\west};
}
\newcommand\macropositivetileBinaryWestNorth{
     \def\south{}
     \def\east{}
     \def\north{\textcolor{blue}{\circled{$1$}}}
     \def\west{\textcolor{blue}{\circled{$1$}}} 
     \draw (\x + 0, \y) --  (\x+1,\y) node [pos=0.5] {\south} -- (\x+1,\y+1) node [pos=0.5] {\east} -- (\x+0,\y+1) node [pos=0.5] {\north} --  (\x + 0, \y)  node [pos=0.5] {\west};
}
\newcommand\macropositivetileBinarySouthEast{
     \def\south{\textcolor{blue}{\circled{$0$}}}
     \def\east{\textcolor{blue}{\circled{$0$}}}
     \def\north{}
     \def\west{} 
     \draw (\x + 0, \y) --  (\x+1,\y) node [pos=0.5] {\south} -- (\x+1,\y+1) node [pos=0.5] {\east} -- (\x+0,\y+1) node [pos=0.5] {\north} --  (\x + 0, \y)  node [pos=0.5] {\west};
}
\newcommand\macropositivetileBinarySouthNorth{
     \def\south{\textcolor{blue}{\circled{$0$}}}
     \def\east{}
     \def\north{\textcolor{blue}{\circled{$1$}}}
     \def\west{} 
     \draw (\x + 0, \y) --  (\x+1,\y) node [pos=0.5] {\south} -- (\x+1,\y+1) node [pos=0.5] {\east} -- (\x+0,\y+1) node [pos=0.5] {\north} --  (\x + 0, \y)  node [pos=0.5] {\west};
}

\newcommand\macronegativetileBinary{
     \def\south{$1$}
     \def\east{$1$}
     \def\north{$0$}
     \def\west{$0$} 
     \draw (\x + 0, \y) --  (\x+1,\y) node [pos=0.5] {\south} -- (\x+1,\y+1) node [pos=0.5] {\east} -- (\x+0,\y+1) node [pos=0.5] {\north} --  (\x + 0, \y)  node [pos=0.5] {\west};
}
\newcommand\macronegativetileBinaryWestEast{
     \def\south{}
     \def\east{\textcolor{blue}{\circled{$1$}}}
     \def\north{}
     \def\west{\textcolor{blue}{\circled{$0$}}}
     \draw (\x + 0, \y) --  (\x+1,\y) node [pos=0.5] {\south} -- (\x+1,\y+1) node [pos=0.5] {\east} -- (\x+0,\y+1) node [pos=0.5] {\north} --  (\x + 0, \y)  node [pos=0.5] {\west};
}
\newcommand\macronegativetileBinaryWestNorth{
     \def\south{}
     \def\east{}
     \def\north{\textcolor{blue}{\circled{$0$}}}
     \def\west{\textcolor{blue}{\circled{$0$}}}
     \draw (\x + 0, \y) --  (\x+1,\y) node [pos=0.5] {\south} -- (\x+1,\y+1) node [pos=0.5] {\east} -- (\x+0,\y+1) node [pos=0.5] {\north} --  (\x + 0, \y)  node [pos=0.5] {\west};
}
\newcommand\macronegativetileBinarySouthEast{
     \def\south{\textcolor{blue}{\circled{$1$}}}
     \def\east{\textcolor{blue}{\circled{$1$}}}
     \def\north{}
     \def\west{} 
     \draw (\x + 0, \y) --  (\x+1,\y) node [pos=0.5,above] {\south} -- (\x+1,\y+1) node [pos=0.5,left] {\east} -- (\x+0,\y+1) node [pos=0.5,above] {\north} --  (\x + 0, \y)  node [pos=0.5,left] {\west};
}

\newcommand\macronegativetileBinarySouthNorth{
     \def\south{\textcolor{blue}{\circled{$1$}}}
     \def\east{}
     \def\north{\textcolor{blue}{\circled{$0$}}}
     \def\west{} 
     \draw (\x + 0, \y) --  (\x+1,\y) node [pos=0.5] {\south} -- (\x+1,\y+1) node [pos=0.5] {\east} -- (\x+0,\y+1) node [pos=0.5] {\north} --  (\x + 0, \y)  node [pos=0.5] {\west};
}
\newcommand\macronegativetileBinaryWest{ % for a last tile
     \def\south{}
     \def\east{}
     \def\north{}
     \def\west{\textcolor{blue}{\circled{$0$}}}
     \draw (\x + 0, \y) --  (\x+1,\y) node [pos=0.5] {\south} -- (\x+1,\y+1) node [pos=0.5] {\east} -- (\x+0,\y+1) node [pos=0.5] {\north} --  (\x + 0, \y)  node [pos=0.5] {\west};
}

% begin snake graph 1011101100 or 2,3,1,2,3 (with 1 0 signs in interior)
 \begin{tikzpicture}[scale=0.65,font=\scriptsize]
 \foreach \x in {0,1,...,2} % bottom row 
 {
\def\y{0}
     \ifthenelse{\x = 0}{\macropositivetileBinaryWestEast}{}
     \ifthenelse{\x = 1}{\macronegativetileBinaryWestEast}{}
     \ifthenelse{\x =2}{\macropositivetileBinaryWestNorth}{}
 }

\def\y{1}
 \foreach \x in {2,...,5} % second from bottom row
 {
     \ifthenelse{\x = 2}
     {\macrotile{} \draw (\x + 0.5, \y+0.5) node {\boxed{$d$}}; }{}
     \ifthenelse{\x = 3}
     {\macropositivetileBinaryEast}{}
     \ifthenelse{\x = 4}
     {\macronegativetileBinaryWestEast}{}
     \ifthenelse{\x = 5}
     {\macropositivetileBinaryWestNorth}{}
 } 
 
\def\y{2}
 \foreach \x in {5} % third from bottom row
 {
     \macronegativetileBinarySouthNorth
 }  

\def\y{3}
\foreach \x in {5,6} % fourth from bottom row
{
     \ifthenelse{\x = 5}{\macropositivetileBinarySouthEast}{}
     \ifthenelse{\x = 6}{\macronegativetileBinaryWest \draw (\x + 0.5, \y+0.5) node {\boxed{$j$}};}{}
 } 
 
\end{tikzpicture}
$\square_{\textcolor{blue}{1011}} = \boxed{d}$ and $\square_{\textcolor{blue}{01100}} = \boxed{j}$

} % end frame



\frame{
\frametitle{Bijection from subwords to perfect matchings (G.)}
\begin{itemize}
\item 
Let $P_{\text{min}}$ be the perfect matching which contains the first south edge and only boundary edges.
%!TEX root = main.tex
\def\myscale{0.27}
\newcommand\macropositivetile{
     \def\south{}
     \def\east{}
     \def\north{}
     \def\west{} 
     \draw (\x + 0, \y) --  (\x+1,\y) 
     %node [pos=0.5,above] {\south} 
     -- (\x+1,\y+1) 
     %node [pos=0.5,left] 
     {\east} -- (\x+0,\y+1) 
     %node [pos=0.5,above] 
     {\north} --  (\x + 0, \y)  
     %node [pos=0.5,left] {\west}
     ;
}
\newcommand\macronegativetile{
     \def\south{}
     \def\east{}
     \def\north{}
     \def\west{} 
     \draw (\x + 0, \y) --  (\x+1,\y) 
     %node [pos=0.5,above] 
     {\south} -- (\x+1,\y+1) 
     %node [pos=0.5,left] 
     {\east} -- (\x+0,\y+1) 
     %node [pos=0.5,above] 
     {\north} --  (\x + 0, \y)  
     %node [pos=0.5,left] {\west}
     ;
}
\newcommand\matchingcolor{2pt}
% begin snake graph 1011101100 or 2,3,1,2,3 with minimal matching
 \begin{tikzpicture}[scale=\myscale,font=\tiny]
 \foreach \x in {0,1,...,2} % bottom row 
 {
%     \draw (\x + 0, 0) -- (\x+1,0) -- (\x+1,1)-- (\x+0,1) -- cycle;
\def\y{0}
     \ifthenelse{\x = 0 \OR \x =2}
     {
     \macropositivetile
     }{
     \macronegativetile
     }
     \ifthenelse{\x = 0 \OR \x =2}{\draw[line width=\matchingcolor] (\x + 0, \y+0) -- (\x+1,\y+0);}{} % south
     \ifthenelse{\x = 0}{\draw[line width=\matchingcolor] (\x + \y+0, \y+1) -- (\x+1,\y+1);}{} % north
 }

\def\y{1}
 \foreach \x in {2,...,5} % second from bottom row
 {
     %\draw (\x + 0, \y) -- (\x+1,\y) -- (\x+1,\y+1)-- (\x+0,\y+1) -- cycle;
     \ifthenelse{\x = 3 \OR \x =5}
     {\macropositivetile}{\macronegativetile}
     \ifthenelse{\x = 2}{\draw[line width=\matchingcolor] (\x+0, \y+0) -- (\x, \y+1);}{} % west
     \ifthenelse{\x = 3 \OR \x =5}{\draw[line width=\matchingcolor] (\x+0, \y+0) -- (\x+1, \y+0);}{} % south
     \ifthenelse{\x = 3}{\draw[line width=\matchingcolor] (\x+0, \y+1) -- (\x+1,\y+1);}{} % north
 } 
 
\def\y{2}
 \foreach \x in {5} % third from bottom row
 {
     %\draw (\x + 0, \y) -- (\x+1,\y) -- (\x+1,\y+1)-- (\x+0,\y+1) -- cycle;
     \macronegativetile
     \ifthenelse{\x = 5}{\draw[line width=\matchingcolor] (\x+0, \y+0) -- (\x, \y+1);}{} % west
     \ifthenelse{\x = 5}{\draw[line width=\matchingcolor] (\x+1, \y) -- (\x+1,\y+1);}{} % east
 }  

\def\y{3}
\foreach \x in {5,6} % fourth from bottom row
{
     %\draw (\x + 0, \y) -- (\x+1,\y) -- (\x+1,\y+1)-- (\x+0,\y+1) -- cycle;
     \ifthenelse{\x = 5}
     {\macropositivetile}{\macronegativetile}
     \ifthenelse{\x = 5}{\draw[line width=\matchingcolor] (\x+0, \y+1) -- (\x+1,\y+1);}{} % north
     \ifthenelse{\x = 6}{\draw[line width=\matchingcolor] (\x+1, \y) -- (\x+1,\y+1);}{} % east
 } 
\end{tikzpicture} % end of minimal matching

\item Let $fil(\square_L)$ be 
the minimal connected sequence of tiles such that $\square_L \in fil(\square_L)$ and the edges bounding $fil(\square_L)$ not in $P_{\text{min}}$ forms a perfect matching of $fil(\square_L)$.

\item Let $\displaystyle fil(s)= \bigcup_{L} fil(\square_L)$.
Let $pm(s) = \{$edges bounding $fil(s)\} \ominus P_{\text{min}}$. 
% symmetric difference
\end{itemize}
%!TEX root = main.tex
\def\macrofilledtile{
\draw [fill=lightgray,opacity=0.99]  (\x + 0, \y) rectangle (\x+1,\y+1);
}
\def\macropositivetile{
     \def\south{}
     \def\east{}
     \def\north{}
     \def\west{} 
     \draw (\x + 0, \y) --  (\x+1,\y) node [pos=0.5,above] {\south} -- (\x+1,\y+1) node [pos=0.5,left] {\east} -- (\x+0,\y+1) node [pos=0.5,above] {\north} --  (\x + 0, \y)  node [pos=0.5,left] {\west};
}
\def\macronegativetile{
     \def\south{}
     \def\east{}
     \def\north{}
     \def\west{} 
     \draw (\x + 0, \y) --  (\x+1,\y) node [pos=0.5,above] {\south} -- (\x+1,\y+1) node [pos=0.5,left] {\east} -- (\x+0,\y+1) node [pos=0.5,above] {\north} --  (\x + 0, \y)  node [pos=0.5,left] {\west};
}
\def\matchingcolor{2pt}
% begin snake graph 1011101100 or 2,3,1,2,3 with matching generated by d and k
 \begin{tikzpicture}[scale=0.65,font=\scriptsize]
 \foreach \x in {0,1,...,2} % bottom row 
 {
%     \draw (\x + 0, 0) -- (\x+1,0) -- (\x+1,1)-- (\x+0,1) -- cycle;
\def\y{0}
     \ifthenelse{\x = 0 \OR \x =2}
     {
     \macropositivetile
     }{
     \macronegativetile
     }
     \ifthenelse{\x = 0 \OR \x =2}{\draw[line width=\matchingcolor] (\x + 0, \y+0) -- (\x+1,\y+0);}{} % south
     \ifthenelse{\x = 0}{\draw[line width=\matchingcolor] (\x + \y+0, \y+1) -- (\x+1,\y+1);}{} % north
 }
\def\y{1}
 \foreach \x in {2,...,5} % second from bottom row
 {
     \ifthenelse{\x = 2 \OR \x = 3}{\macrofilledtile}{}
     \ifthenelse{\x = 2}
     {\draw (\x + 0.5, \y+0.5) node {\boxed{$d$}}; }{}
     \ifthenelse{\x = 3}
     {\draw (\x + 0.5, \y+0.5) node {{$e$}}; }{}
     %\draw (\x + 0, \y) -- (\x+1,\y) -- (\x+1,\y+1)-- (\x+0,\y+1) -- cycle;
     \ifthenelse{\x = 3 \OR \x =5}
     {\macropositivetile}{\macronegativetile}
     \ifthenelse{\x = 2}{\draw[line width=\matchingcolor] (\x+0, \y+0) -- (\x, \y+1);}{} % west
     \ifthenelse{\x = 2}{\draw[red, densely dashed, line width=1.5*\matchingcolor] (\x+0, \y+0) -- (\x+1, \y+0);}{} % south
     \ifthenelse{\x = 2}{\draw[red, densely dashed, line width=1.5*\matchingcolor] (\x+0, \y+1) -- (\x+1,\y+1);}{} % north
      \ifthenelse{\x = 3}{\draw[red, densely dashed, line width=1.5*\matchingcolor] (\x+1, \y) -- (\x+1,\y+1);}{} % east
      \ifthenelse{\x = 3}{\draw[line width=\matchingcolor] (\x+0, \y+0) -- (\x+1, \y+0);}{} % south
     \ifthenelse{\x = 3}{\draw[line width=\matchingcolor] (\x+0, \y+1) -- (\x+1,\y+1);}{} % north
      \ifthenelse{\x =5}{\draw[line width=\matchingcolor] (\x+0, \y+0) -- (\x+1, \y+0);}{} % south
 } 
\def\y{2}
 \foreach \x in {5} % third from bottom row
 {
     \macrofilledtile
     %\draw (\x + 0, \y) -- (\x+1,\y) -- (\x+1,\y+1)-- (\x+0,\y+1) -- cycle;
     \macronegativetile
     \ifthenelse{\x = 5}{\draw[line width=\matchingcolor] (\x+0, \y+0) -- (\x, \y+1);}{} % west
     \ifthenelse{\x = 5}{\draw[line width=\matchingcolor] (\x+1, \y) -- (\x+1,\y+1);}{} % east
     \ifthenelse{\x = 5}{\draw[red,densely dashed,line width=1.5*\matchingcolor] (\x+0, \y+0) -- (\x+1, \y+0);}{} % south
     \draw (\x + 0.5, \y+0.5) node {{$h$}};
 }  
\def\y{3}
\foreach \x in {5,6} % fourth from bottom row
{
    \macrofilledtile
     %\draw (\x + 0, \y) -- (\x+1,\y) -- (\x+1,\y+1)-- (\x+0,\y+1) -- cycle;
     \ifthenelse{\x = 5}
     {\macropositivetile}{\macronegativetile}
     \ifthenelse{\x = 5}{\draw[line width=\matchingcolor] (\x+0, \y+1) -- (\x+1,\y+1);}{} % north
     \ifthenelse{\x = 5}{\draw[red,densely dashed,line width=1.5*\matchingcolor] (\x+0, \y+0) -- (\x, \y+1);}{} % west

     \ifthenelse{\x = 6}{\draw[line width=\matchingcolor] (\x+1, \y) -- (\x+1,\y+1);}{} % east
     \ifthenelse{\x = 6}{\draw[red,densely dashed,line width=1.5*\matchingcolor] (\x+0, \y+0) -- (\x+1, \y+0);}{} % south
     \ifthenelse{\x = 6}{\draw[red,densely dashed,line width=1.5*\matchingcolor] (\x+0, \y+1) -- (\x+1,\y+1);}{} % north
     \ifthenelse{\x = 5}{\draw (\x + 0.5, \y+0.5) node {{$i$}}; }{}
     \ifthenelse{\x = 6}{\draw (\x + 0.5, \y+0.5) node {\boxed{$j$}}; }{}
 } 
\end{tikzpicture} % end of matching subword with 1 missing
%!TEX root = main.tex
\def\macrofilledtile{
\draw [fill=lightgray,opacity=0.99]  (\x + 0, \y) rectangle (\x+1,\y+1);
}
\def\macropositivetile{
     \def\south{}
     \def\east{}
     \def\north{}
     \def\west{} 
     \draw (\x + 0, \y) --  (\x+1,\y) node [pos=0.5,above] {\south} -- (\x+1,\y+1) node [pos=0.5,left] {\east} -- (\x+0,\y+1) node [pos=0.5,above] {\north} --  (\x + 0, \y)  node [pos=0.5,left] {\west};
}
\def\macronegativetile{
     \def\south{}
     \def\east{}
     \def\north{}
     \def\west{} 
     \draw (\x + 0, \y) --  (\x+1,\y) node [pos=0.5,above] {\south} -- (\x+1,\y+1) node [pos=0.5,left] {\east} -- (\x+0,\y+1) node [pos=0.5,above] {\north} --  (\x + 0, \y)  node [pos=0.5,left] {\west};
}
\def\matchingcolor{2pt}
% begin snake graph 1011101100 or 2,3,1,2,3 with matching generated by d and k
 \begin{tikzpicture}[scale=0.65,font=\scriptsize]
 \foreach \x in {0,1,...,2} % bottom row 
 {
%     \draw (\x + 0, 0) -- (\x+1,0) -- (\x+1,1)-- (\x+0,1) -- cycle;
\def\y{0}
     \ifthenelse{\x = 0 \OR \x =2}
     {
     \macropositivetile
     }{
     \macronegativetile
     }
     \ifthenelse{\x = 0 \OR \x =2}{\draw[line width=\matchingcolor] (\x + 0, \y+0) -- (\x+1,\y+0);}{} % south
     \ifthenelse{\x = 0}{\draw[line width=\matchingcolor] (\x + \y+0, \y+1) -- (\x+1,\y+1);}{} % north
 }
\def\y{1}
 \foreach \x in {2,...,5} % second from bottom row
 {
     \ifthenelse{\x = 2 \OR \x = 3}{\macrofilledtile}{}
     %\ifthenelse{\x = 2}{\draw (\x + 0.5, \y+0.5) node {\boxed{$d$}}; }{}
     %\ifthenelse{\x = 3}{\draw (\x + 0.5, \y+0.5) node {{$e$}}; }{}
     %\draw (\x + 0, \y) -- (\x+1,\y) -- (\x+1,\y+1)-- (\x+0,\y+1) -- cycle;
     \ifthenelse{\x = 3 \OR \x =5}{\macropositivetile}{\macronegativetile}
     %\ifthenelse{\x = 2}{\draw[line width=\matchingcolor] (\x+0, \y+0) -- (\x, \y+1);}{} % west
     \ifthenelse{\x = 2}{\draw[red, %densely dashed, 
     line width=1.5*\matchingcolor] (\x+0, \y+0) -- (\x+1, \y+0);}{} % south
     \ifthenelse{\x = 2}{\draw[red, %densely dashed, 
     line width=1.5*\matchingcolor] (\x+0, \y+1) -- (\x+1,\y+1);}{} % north
      \ifthenelse{\x = 3}{\draw[red, %densely dashed,
      line width=1.5*\matchingcolor] (\x+1, \y) -- (\x+1,\y+1);}{} % east
      %\ifthenelse{\x = 3}{\draw[line width=\matchingcolor] (\x+0, \y+0) -- (\x+1, \y+0);}{} % south
     %\ifthenelse{\x = 3}{\draw[line width=\matchingcolor] (\x+0, \y+1) -- (\x+1,\y+1);}{} % north
      \ifthenelse{\x =5}{\draw[line width=\matchingcolor] (\x+0, \y+0) -- (\x+1, \y+0);}{} % south
 } 
\def\y{2}
 \foreach \x in {5} % third from bottom row
 {
     \macrofilledtile
     %\draw (\x + 0, \y) -- (\x+1,\y) -- (\x+1,\y+1)-- (\x+0,\y+1) -- cycle;
     \macronegativetile
     %\ifthenelse{\x = 5}{\draw[line width=\matchingcolor] (\x+0, \y+0) -- (\x, \y+1);}{} % west
     %\ifthenelse{\x = 5}{\draw[line width=\matchingcolor] (\x+1, \y) -- (\x+1,\y+1);}{} % east
     \ifthenelse{\x = 5}{\draw[red,%densely dashed,
     line width=1.5*\matchingcolor] (\x+0, \y+0) -- (\x+1, \y+0);}{} % south
     %\draw (\x + 0.5, \y+0.5) node {{$h$}};
 }  
\def\y{3}
\foreach \x in {5,6} % fourth from bottom row
{
    \macrofilledtile
     %\draw (\x + 0, \y) -- (\x+1,\y) -- (\x+1,\y+1)-- (\x+0,\y+1) -- cycle;
     \ifthenelse{\x = 5}
     {\macropositivetile}{\macronegativetile}
     %\ifthenelse{\x = 5}{\draw[line width=\matchingcolor] (\x+0, \y+1) -- (\x+1,\y+1);}{} % north
     \ifthenelse{\x = 5}{\draw[red,%densely dashed,
     line width=1.5*\matchingcolor] (\x+0, \y+0) -- (\x, \y+1);}{} % west

     %\ifthenelse{\x = 6}{\draw[line width=\matchingcolor] (\x+1, \y) -- (\x+1,\y+1);}{} % east
     \ifthenelse{\x = 6}{\draw[red,%densely dashed,
     line width=1.5*\matchingcolor] (\x+0, \y+0) -- (\x+1, \y+0);}{} % south
     \ifthenelse{\x = 6}{\draw[red,%densely dashed,
     line width=1.5*\matchingcolor] (\x+0, \y+1) -- (\x+1,\y+1);}{} % north
     %\ifthenelse{\x = 5}{\draw (\x + 0.5, \y+0.5) node {{$i$}}; }{}
     %\ifthenelse{\x = 6}{\draw (\x + 0.5, \y+0.5) node {\boxed{$j$}}; }{}
 } 
\end{tikzpicture} % end of matching subword with 1 missing

} % end frame


\frame{
%Another example with $s=11010$ and 
%!TEX root = main.tex
\def\myscale{0.75}
\def\myscalematching{0.65}
\frametitle{Another example: From $11010$ to a perfect matching}
\newcommand\macrotile[1]{
     \def\south{}
     \def\east{}
     \def\north{}
     \def\west{} 
     \draw (\x + 0, \y) --  (\x+1,\y) node [pos=0.5,above] {\south} -- (\x+1,\y+1) node [pos=0.5,left] {\east} -- (\x+0,\y+1) node [pos=0.5,above] {\north} --  (\x + 0, \y)  node [pos=0.5,left] {\west};
}
%\newcommand\macropositivetileBinary{
%     \def\south{$0$}
%     \def\east{$0$}
%     \def\north{$1$}
%     \def\west{$1$} 
%     \draw (\x + 0, \y) --  (\x+1,\y) node [pos=0.5,above] {\south} -- (\x+1,\y+1) node [pos=0.5,left] {\east} -- (\x+0,\y+1) node [pos=0.5,above] {\north} --  (\x + 0, \y)  node [pos=0.5,left] {\west};
%}
\newcommand\macropositivetileBinaryWestEast{
     \def\south{}
     \def\east{\textcolor{blue}{\circled{$0$}}}
     \def\north{}
     \def\west{\textcolor{blue}{\circled{$1$}}}
     \draw (\x + 0, \y) --  (\x+1,\y) node [pos=0.5] {\south} -- (\x+1,\y+1) node [pos=0.5] {\east} -- (\x+0,\y+1) node [pos=0.5] {\north} --  (\x + 0, \y)  node [pos=0.5] {\west};
}
\newcommand\macropositivetileBinaryWest{
     \def\south{}
     \def\east{{$0$}}
     \def\north{}
     \def\west{\textcolor{blue}{\circled{$1$}}}
     \draw (\x + 0, \y) --  (\x+1,\y) node [pos=0.5] {\south} -- (\x+1,\y+1) node [pos=0.5] {\east} -- (\x+0,\y+1) node [pos=0.5] {\north} --  (\x + 0, \y)  node [pos=0.5] {\west};
}
\newcommand\macropositivetileBinaryEast{
     \def\south{}
     \def\east{\textcolor{blue}{\circled{$0$}}}
     \def\north{}
     \def\west{{$1$}} 
     \draw (\x + 0, \y) --  (\x+1,\y) node [pos=0.5] {\south} -- (\x+1,\y+1) node [pos=0.5] {\east} -- (\x+0,\y+1) node [pos=0.5] {\north} --  (\x + 0, \y)  node [pos=0.5] {\west};
}
\newcommand\macropositivetileBinaryWestNorth{
     \def\south{}
     \def\east{}
     \def\north{\textcolor{blue}{\circled{$1$}}}
     \def\west{\textcolor{blue}{\circled{$1$}}} 
     \draw (\x + 0, \y) --  (\x+1,\y) node [pos=0.5] {\south} -- (\x+1,\y+1) node [pos=0.5] {\east} -- (\x+0,\y+1) node [pos=0.5] {\north} --  (\x + 0, \y)  node [pos=0.5] {\west};
}
\newcommand\macropositivetileBinarySouthEast{
     \def\south{\textcolor{blue}{\circled{$0$}}}
     \def\east{\textcolor{blue}{\circled{$0$}}}
     \def\north{}
     \def\west{} 
     \draw (\x + 0, \y) --  (\x+1,\y) node [pos=0.5] {\south} -- (\x+1,\y+1) node [pos=0.5] {\east} -- (\x+0,\y+1) node [pos=0.5] {\north} --  (\x + 0, \y)  node [pos=0.5] {\west};
}
\newcommand\macropositivetileBinarySouth{
     \def\south{\textcolor{blue}{\circled{$0$}}}
     \def\east{}%{\textcolor{blue}{\circled{$0$}}}
     \def\north{}
     \def\west{} 
     \draw (\x + 0, \y) --  (\x+1,\y) node [pos=0.5] {\south} -- (\x+1,\y+1) node [pos=0.5] {\east} -- (\x+0,\y+1) node [pos=0.5] {\north} --  (\x + 0, \y)  node [pos=0.5] {\west};
}
\newcommand\macropositivetileBinarySouthNorth{
     \def\south{\textcolor{blue}{\circled{$0$}}}
     \def\east{}
     \def\north{\textcolor{blue}{\circled{$1$}}}
     \def\west{} 
     \draw (\x + 0, \y) --  (\x+1,\y) node [pos=0.5] {\south} -- (\x+1,\y+1) node [pos=0.5] {\east} -- (\x+0,\y+1) node [pos=0.5] {\north} --  (\x + 0, \y)  node [pos=0.5] {\west};
}

\newcommand\macronegativetileBinary{
     \def\south{$1$}
     \def\east{$1$}
     \def\north{$0$}
     \def\west{$0$} 
     \draw (\x + 0, \y) --  (\x+1,\y) node [pos=0.5] {\south} -- (\x+1,\y+1) node [pos=0.5] {\east} -- (\x+0,\y+1) node [pos=0.5] {\north} --  (\x + 0, \y)  node [pos=0.5] {\west};
}
\newcommand\macronegativetileBinaryWestEast{
     \def\south{}
     \def\east{\textcolor{blue}{\circled{$1$}}}
     \def\north{}
     \def\west{\textcolor{blue}{\circled{$0$}}}
     \draw (\x + 0, \y) --  (\x+1,\y) node [pos=0.5] {\south} -- (\x+1,\y+1) node [pos=0.5] {\east} -- (\x+0,\y+1) node [pos=0.5] {\north} --  (\x + 0, \y)  node [pos=0.5] {\west};
}
\newcommand\macronegativetileBinaryEast{
     \def\south{}
     \def\east{\textcolor{blue}{\circled{$1$}}}
     \def\north{}
     \def\west{{$0$}}
     \draw (\x + 0, \y) --  (\x+1,\y) node [pos=0.5] {\south} -- (\x+1,\y+1) node [pos=0.5] {\east} -- (\x+0,\y+1) node [pos=0.5] {\north} --  (\x + 0, \y)  node [pos=0.5] {\west};
}
\newcommand\macronegativetileBinaryWestNorth{
     \def\south{}
     \def\east{}
     \def\north{\textcolor{blue}{\circled{$0$}}}
     \def\west{\textcolor{blue}{\circled{$0$}}}
     \draw (\x + 0, \y) --  (\x+1,\y) node [pos=0.5] {\south} -- (\x+1,\y+1) node [pos=0.5] {\east} -- (\x+0,\y+1) node [pos=0.5] {\north} --  (\x + 0, \y)  node [pos=0.5] {\west};
}
\newcommand\macronegativetileBinarySouthEast{
     \def\south{\textcolor{blue}{\circled{$1$}}}
     \def\east{\textcolor{blue}{\circled{$1$}}}
     \def\north{}
     \def\west{} 
     \draw (\x + 0, \y) --  (\x+1,\y) node [pos=0.5,above] {\south} -- (\x+1,\y+1) node [pos=0.5,left] {\east} -- (\x+0,\y+1) node [pos=0.5,above] {\north} --  (\x + 0, \y)  node [pos=0.5,left] {\west};
}
\newcommand\macronegativetileBinarySouthEastNotcircled{
     \def\south{\textcolor{black}{{$1$}}}
     \def\east{\textcolor{black}{{$1$}}}
     \def\north{}
     \def\west{} 
     \draw (\x + 0, \y) --  (\x+1,\y) node [pos=0.5,above] {\south} -- (\x+1,\y+1) node [pos=0.5] {\east} -- (\x+0,\y+1) node [pos=0.5,above] {\north} --  (\x + 0, \y)  node [pos=0.5] {\west};
}

\newcommand\macronegativetileBinarySouthNorth{
     \def\south{\textcolor{blue}{\circled{$1$}}}
     \def\east{}
     \def\north{\textcolor{blue}{\circled{$0$}}}
     \def\west{} 
     \draw (\x + 0, \y) --  (\x+1,\y) node [pos=0.5] {\south} -- (\x+1,\y+1) node [pos=0.5] {\east} -- (\x+0,\y+1) node [pos=0.5] {\north} --  (\x + 0, \y)  node [pos=0.5] {\west};
}
\newcommand\macronegativetileBinaryNorthNotcircled{
     \def\south{\textcolor{black}{{$1$}}}
     \def\east{}
     \def\north{\textcolor{black}{{$0$}}}
     \def\west{} 
     \draw (\x + 0, \y) --  (\x+1,\y) node [pos=0.5] {\south} -- (\x+1,\y+1) node [pos=0.5] {\east} -- (\x+0,\y+1) node [pos=0.5] {\north} --  (\x + 0, \y)  node [pos=0.5] {\west};
}
\newcommand\macronegativetileBinaryWest{ % for a last tile
     \def\south{}
     \def\east{}
     \def\north{}
     \def\west{\textcolor{blue}{\circled{$0$}}}
     \draw (\x + 0, \y) --  (\x+1,\y) node [pos=0.5] {\south} -- (\x+1,\y+1) node [pos=0.5] {\east} -- (\x+0,\y+1) node [pos=0.5] {\north} --  (\x + 0, \y)  node [pos=0.5] {\west};
}
\newcommand\macronegativetileBinaryWestNotcircled{ % for a last tile
     \def\south{}
     \def\east{}
     \def\north{}
     \def\west{\textcolor{black}{{$0$}}}
     \draw (\x + 0, \y) --  (\x+1,\y) node [pos=0.5] {\south} -- (\x+1,\y+1) node [pos=0.5] {\east} -- (\x+0,\y+1) node [pos=0.5] {\north} --  (\x + 0, \y)  node [pos=0.5] {\west};
}

% begin snake graph 1011101100 or 2,3,1,2,3 (with 1 0 signs in interior)
 \begin{tikzpicture}[scale=\myscale,font=\scriptsize]
 \foreach \x in {0,1,...,2} % bottom row 
 {
\def\y{0}
     \ifthenelse{\x = 0}{\macropositivetileBinaryWest      \draw (\x + 0.5, \y+0.5) node {\boxed{$a$}};}{}
     \ifthenelse{\x = 1}{\macronegativetileBinaryEast}{}
     \ifthenelse{\x =2}{\macrotile{} %\macropositivetileBinaryWestNorth 
     \draw (\x + 0.5, \y+0.5) node {\boxed{$c$}};}{}
 }

\def\y{1}
 \foreach \x in {2,...,5} % second from bottom row
 {
     \ifthenelse{\x = 2}
     {\macronegativetileBinarySouthEastNotcircled }{}
     \ifthenelse{\x = 3}
     {\macropositivetileBinaryEast}{}
     \ifthenelse{\x = 4}
     {\macronegativetileBinaryWestEast}{}
     \ifthenelse{\x = 5}
     {\macrotile{} %\macropositivetileBinaryWestNorth 
     \draw (\x + 0.5, \y+0.5) node {\boxed{$g$}};
     }{}
 } 
 
\def\y{2}
 \foreach \x in {5} % third from bottom row
 {
     \macronegativetileBinaryNorthNotcircled{} %\macronegativetileBinarySouthNorth
 }  

\def\y{3}
\foreach \x in {5,6} % fourth from bottom row
{
     \ifthenelse{\x = 5}{\macropositivetileBinarySouth
     \draw (\x + 0.5, \y+0.5) node {\boxed{$i$}};}{}
     \ifthenelse{\x = 6}{\macronegativetileBinaryWestNotcircled }{}
 } 
 
\end{tikzpicture}

$\square_{\textcolor{blue}{1}} = \boxed{a}$, $\square_{\textcolor{blue}{1}} = \boxed{c}$ , $\square_{\textcolor{blue}{01}} = \boxed{g}$, and $\square_{\textcolor{blue}{0}} = \boxed{i}$.



\def\macrofilledtile{
\draw [fill=lightgray,opacity=0.99]  (\x + 0, \y) rectangle (\x+1,\y+1);
}
\def\macropositivetile{
     \def\south{}
     \def\east{}
     \def\north{}
     \def\west{} 
     \draw (\x + 0, \y) --  (\x+1,\y) node [pos=0.5,above] {\south} -- (\x+1,\y+1) node [pos=0.5,left] {\east} -- (\x+0,\y+1) node [pos=0.5,above] {\north} --  (\x + 0, \y)  node [pos=0.5,left] {\west};
}
\def\macronegativetile{
     \def\south{}
     \def\east{}
     \def\north{}
     \def\west{} 
     \draw (\x + 0, \y) --  (\x+1,\y) node [pos=0.5,above] {\south} -- (\x+1,\y+1) node [pos=0.5,left] {\east} -- (\x+0,\y+1) node [pos=0.5,above] {\north} --  (\x + 0, \y)  node [pos=0.5,left] {\west};
}
\def\matchingcolor{2pt}
% begin snake graph 1011101100 or 2,3,1,2,3 with matching generated by acgi
 \begin{tikzpicture}[scale=\myscalematching,
 font=\footnotesize%\scriptsize
 ]
 \foreach \x in {0,1,...,2} % bottom row 
 {
%     \draw (\x + 0, 0) -- (\x+1,0) -- (\x+1,1)-- (\x+0,1) -- cycle;
\def\y{0}
     \ifthenelse{\x = 0 \OR \x =2}
     {
     \macropositivetile
     }{
     \macronegativetile
     }
     \ifthenelse{\x = 0 \OR \x =2}{\draw[line width=\matchingcolor] (\x + 0, \y+0) -- (\x+1,\y+0);}{} % south
     \ifthenelse{\x = 0}{\macrofilledtile \draw[line width=\matchingcolor] (\x + \y+0, \y+1) -- (\x+1,\y+1); }{} % north
     \ifthenelse{\x = 0}{\draw (\x + 0.5, \y+0.5) node {\boxed{$a$}}; }{}
     \ifthenelse{\x = 2}{ \macrofilledtile\draw (\x + 0.5, \y+0.5) node {\boxed{$c$}}; }{}
     
          \ifthenelse{\x = 0 \OR \x = 2}{\draw[red,densely dashed,line width=1.5*\matchingcolor] (\x+0, \y+0) -- (\x, \y+1);}{} % west
          \ifthenelse{\x = 0 \OR \x = 2}{\draw[red,densely dashed,line width=1.5*\matchingcolor] (\x+1, \y+0) -- (\x+1, \y+1);}{} % east
 }
\def\y{1}
 \foreach \x in {2,...,5} % second from bottom row
 {
     \ifthenelse{\x = 2 \OR \x = 3 \OR \x = 5}{\macrofilledtile}{}
     \ifthenelse{\x = 2}
     {\draw (\x + 0.5, \y+0.5) node {{$d$}}; }{}
     \ifthenelse{\x = 3}
     {\draw (\x + 0.5, \y+0.5) node {{$e$}}; }{}
     %\draw (\x + 0, \y) -- (\x+1,\y) -- (\x+1,\y+1)-- (\x+0,\y+1) -- cycle;
     \ifthenelse{\x = 3 \OR \x =5}
     {\macropositivetile}{\macronegativetile}
     \ifthenelse{\x = 2}{\draw[line width=\matchingcolor] (\x+0, \y+0) -- (\x, \y+1);}{} % west
%     \ifthenelse{\x = 2}{\draw[red, densely dashed, line width=1.5*\matchingcolor] (\x+0, \y+0) -- (\x+1, \y+0);}{} % south
     \ifthenelse{\x = 2}{\draw[red, densely dashed, line width=1.5*\matchingcolor] (\x+0, \y+1) -- (\x+1,\y+1);}{} % north
      \ifthenelse{\x = 3}{\draw[red, densely dashed, line width=1.5*\matchingcolor] (\x+1, \y) -- (\x+1,\y+1);}{} % east
      \ifthenelse{\x = 3}{\draw[line width=\matchingcolor] (\x+0, \y+0) -- (\x+1, \y+0);}{} % south
     \ifthenelse{\x = 3}{\draw[line width=\matchingcolor] (\x+0, \y+1) -- (\x+1,\y+1);}{} % north
      \ifthenelse{\x =5}{\draw[line width=\matchingcolor] (\x+0, \y+0) -- (\x+1, \y+0);}{} % south
      \ifthenelse{\x = 5}{\draw (\x + 0.5, \y+0.5) node {\boxed{$g$}}; }{}
      \ifthenelse{\x = 5}{\draw[red,densely dashed,line width=1.5*\matchingcolor] (\x+0, \y+0) -- (\x, \y+1);}{} % west
          \ifthenelse{\x = 5}{\draw[red,densely dashed,line width=1.5*\matchingcolor] (\x+1, \y+0) -- (\x+1, \y+1);}{} % east
 } 
\def\y{2}
 \foreach \x in {5} % third from bottom row
 {
     \macrofilledtile
     %\draw (\x + 0, \y) -- (\x+1,\y) -- (\x+1,\y+1)-- (\x+0,\y+1) -- cycle;
     \macronegativetile
     \ifthenelse{\x = 5}{\draw[line width=\matchingcolor] (\x+0, \y+0) -- (\x, \y+1);}{} % west
     \ifthenelse{\x = 5}{\draw[line width=\matchingcolor] (\x+1, \y) -- (\x+1,\y+1);}{} % east
%     \ifthenelse{\x = 5}{\draw[red,densely dashed,line width=1.5*\matchingcolor] (\x+0, \y+0) -- (\x+1, \y+0);}{} % south
     \draw (\x + 0.5, \y+0.5) node {{$h$}};
 }  
\def\y{3}
\foreach \x in {5,6} % fourth from bottom row
{
     %\draw (\x + 0, \y) -- (\x+1,\y) -- (\x+1,\y+1)-- (\x+0,\y+1) -- cycle;
     \ifthenelse{\x = 5}
     {\macropositivetile \macrofilledtile}{\macronegativetile}
     \ifthenelse{\x = 5}{\draw[line width=\matchingcolor] (\x+0, \y+1) -- (\x+1,\y+1);}{} % north
     \ifthenelse{\x = 5}{\draw[red,densely dashed,line width=1.5*\matchingcolor] (\x+0, \y+0) -- (\x, \y+1);}{} % west
          \ifthenelse{\x = 5}{\draw[red,densely dashed,line width=1.5*\matchingcolor] (\x+1, \y+0) -- (\x+1, \y+1);}{} % east

     \ifthenelse{\x = 6}{\draw[line width=\matchingcolor] (\x+1, \y) -- (\x+1,\y+1);}{} % east
%     \ifthenelse{\x = 6}{\draw[red,densely dashed,line width=1.5*\matchingcolor] (\x+0, \y+0) -- (\x+1, \y+0);}{} % south
%     \ifthenelse{\x = 6}{\draw[red,densely dashed,line width=1.5*\matchingcolor] (\x+0, \y+1) -- (\x+1,\y+1);}{} % north
     \ifthenelse{\x = 5}{\draw (\x + 0.5, \y+0.5) node {\boxed{$i$}}; }{}
 } 
\end{tikzpicture} % end of matching acgi
% begin snake graph 1011101100 or 2,3,1,2,3 with matching generated by acgi
 \begin{tikzpicture}[scale=\myscalematching,font=\scriptsize]
 \foreach \x in {0,1,...,2} % bottom row 
 {
%     \draw (\x + 0, 0) -- (\x+1,0) -- (\x+1,1)-- (\x+0,1) -- cycle;
\def\y{0}
     \ifthenelse{\x = 0 \OR \x =2}
     {
     \macropositivetile
     }{
     \macronegativetile
     }
%     \ifthenelse{\x = 0 \OR \x =2}{\draw[line width=\matchingcolor] (\x + 0, \y+0) -- (\x+1,\y+0);}{} % south
%     \ifthenelse{\x = 0}{ \draw[line width=\matchingcolor] (\x + \y+0, \y+1) -- (\x+1,\y+1); }{} % north
     \ifthenelse{\x = 0}{\macrofilledtile %\draw (\x + 0.5, \y+0.5) node {\boxed{$a$}};
      }{}
     \ifthenelse{\x = 2}{ \macrofilledtile %\draw (\x + 0.5, \y+0.5) node {\boxed{$c$}}; 
     }{}
     
          \ifthenelse{\x = 0 \OR \x = 2}{\draw[red,line width=1.5*\matchingcolor] (\x+0, \y+0) -- (\x, \y+1);}{} % west
          \ifthenelse{\x = 0 \OR \x = 2}{\draw[red,line width=1.5*\matchingcolor] (\x+1, \y+0) -- (\x+1, \y+1);}{} % east
 }
\def\y{1}
 \foreach \x in {2,...,5} % second from bottom row
 {
     \ifthenelse{\x = 2 \OR \x = 3 \OR \x = 5}{\macrofilledtile}{}
     %\ifthenelse{\x = 2}{\draw (\x + 0.5, \y+0.5) node {{$d$}}; }{}
     %\ifthenelse{\x = 3}{\draw (\x + 0.5, \y+0.5) node {{$e$}}; }{}
     %\draw (\x + 0, \y) -- (\x+1,\y) -- (\x+1,\y+1)-- (\x+0,\y+1) -- cycle;
     \ifthenelse{\x = 3 \OR \x =5}
     {\macropositivetile}{\macronegativetile}
%     \ifthenelse{\x = 2}{\draw[line width=\matchingcolor] (\x+0, \y+0) -- (\x, \y+1);}{} % west
%     \ifthenelse{\x = 2}{\draw[red, densely dashed, line width=1.5*\matchingcolor] (\x+0, \y+0) -- (\x+1, \y+0);}{} % south
     \ifthenelse{\x = 2}{\draw[red, line width=1.5*\matchingcolor] (\x+0, \y+1) -- (\x+1,\y+1);}{} % north
      \ifthenelse{\x = 3}{\draw[red, line width=1.5*\matchingcolor] (\x+1, \y) -- (\x+1,\y+1);}{} % east
%      \ifthenelse{\x = 3}{\draw[line width=\matchingcolor] (\x+0, \y+0) -- (\x+1, \y+0);}{} % south
%     \ifthenelse{\x = 3}{\draw[line width=\matchingcolor] (\x+0, \y+1) -- (\x+1,\y+1);}{} % north
%      \ifthenelse{\x =5}{\draw[line width=\matchingcolor] (\x+0, \y+0) -- (\x+1, \y+0);}{} % south
      %\ifthenelse{\x = 5}{\draw (\x + 0.5, \y+0.5) node {\boxed{$g$}}; }{}
      \ifthenelse{\x = 5}{\draw[red,line width=1.5*\matchingcolor] (\x+0, \y+0) -- (\x, \y+1);}{} % west
          \ifthenelse{\x = 5}{\draw[red,line width=1.5*\matchingcolor] (\x+1, \y+0) -- (\x+1, \y+1);}{} % east
 } 
\def\y{2}
 \foreach \x in {5} % third from bottom row
 {
     \macrofilledtile
     %\draw (\x + 0, \y) -- (\x+1,\y) -- (\x+1,\y+1)-- (\x+0,\y+1) -- cycle;
     \macronegativetile
%     \ifthenelse{\x = 5}{\draw[line width=\matchingcolor] (\x+0, \y+0) -- (\x, \y+1);}{} % west
%     \ifthenelse{\x = 5}{\draw[line width=\matchingcolor] (\x+1, \y) -- (\x+1,\y+1);}{} % east
%     \ifthenelse{\x = 5}{\draw[red,densely dashed,line width=1.5*\matchingcolor] (\x+0, \y+0) -- (\x+1, \y+0);}{} % south
    ;% \draw (\x + 0.5, \y+0.5) node {{$h$}};
 }  
\def\y{3}
\foreach \x in {5,6} % fourth from bottom row
{
     %\draw (\x + 0, \y) -- (\x+1,\y) -- (\x+1,\y+1)-- (\x+0,\y+1) -- cycle;
     \ifthenelse{\x = 5}
     {\macropositivetile \macrofilledtile}{\macronegativetile}
%     \ifthenelse{\x = 5}{\draw[line width=\matchingcolor] (\x+0, \y+1) -- (\x+1,\y+1);}{} % north
     \ifthenelse{\x = 5}{\draw[red,line width=1.5*\matchingcolor] (\x+0, \y+0) -- (\x, \y+1);}{} % west
          \ifthenelse{\x = 5}{\draw[red,line width=1.5*\matchingcolor] (\x+1, \y+0) -- (\x+1, \y+1);}{} % east

     \ifthenelse{\x = 6}{\draw[line width=\matchingcolor] (\x+1, \y) -- (\x+1,\y+1);}{} % east
%     \ifthenelse{\x = 6}{\draw[red,densely dashed,line width=1.5*\matchingcolor] (\x+0, \y+0) -- (\x+1, \y+0);}{} % south
%     \ifthenelse{\x = 6}{\draw[red,densely dashed,line width=1.5*\matchingcolor] (\x+0, \y+1) -- (\x+1,\y+1);}{} % north
     %\ifthenelse{\x = 5}{\draw (\x + 0.5, \y+0.5) node {\boxed{$i$}}; }{}
 } 
\end{tikzpicture} % end of matching acgi

}

\backupbegin

\frame{
\vfill
{\Huge Thank you}
\vfill 
Comments and suggestions are welcome
}


\frame{
\frametitle{Example: arc to poset}
%!TEX root = main.tex
\def\mypolygonscale{0.9}
\def\myyscale{1.2}
\def\myxscale{1.0}
\definecolor{mygreen}{cmyk}{0.5,0,0.5,0.5}
\begin{tikzpicture}[scale = \mypolygonscale]
%\tikzstyle{every node} = [font = \small]
\foreach \x in {0}
{
    \foreach \y in {-8}
    {
			\fill (\x,\y+1.75) circle (.05);
			%\fill (\x,\y+1.75) node [above] {\tiny{$1$}};
			\fill(\x+1.5158,\y+.875) circle (.05);
			%\fill (\x+1.5158,\y+.875) node [right] {\tiny{$2 $}};
			\fill (\x+1.5158,\y-.875) circle (.05);
			%\fill (\x+1.5158,\y-.875) node [right] {\tiny{$3$}};
			\fill(\x,\y-1.75) circle (.05);
			%\fill (\x,\y-1.75) node [below] {\tiny{$4$}};
			\fill(\x-1.5158,\y-.875) circle (.05);
			%\fill (\x-1.5158,\y-.875) node [left] {\tiny{$5$}};
			\fill (\x-1.5158,\y+.875) circle (.05);
			%\fill (\x-1.5158,\y+.875) node [left] {\tiny{$6$}};
		
			\draw [] (\x,\y+1.75)--(\x+1.5158,\y+.875); %v1 to v2
			\draw [] (\x+1.5158,\y+.875) -- (\x+1.5158,\y-.875); %v3 to v2
			\draw [] (\x+1.5158,\y-.875)--(\x,\y-1.75); %v4 to v3
			\draw [] (\x,\y-1.75) -- (\x-1.5158,\y-.875); %v5 to v4
			\draw [] (\x-1.5158,\y-.875)--(\x-1.5158,\y+.875); %v6 to v5
			\draw [] (\x-1.5158,\y+.875) -- (\x,\y+1.75); %v1 to v6
		
\draw [red] (\x,\y+1.75) -- (\x-1.5158,\y-.875) node[pos=0.8] {$x_a$};
%\draw[->,densely dotted, thick]  (\x,\y) -- (\x-1.5158*0.5,\y + 1.75*.5 - .875*.5); 
%\draw ((\x+\x-1.5158)*0.5,(\y+\y1.75 - .875)*.5) -- (\x,\y+1.75); 
\draw [blue] (\x,\y+1.75) -- (\x,\y-1.75) node[pos=0.6] {$x_b$};
%\draw[densely dashed, ultra thick,rounded corners=2mm]  (\x+1.5158,\y+.875) -- (\x+1.5158*0.5,\y + 1.75*.5 - .875*.5) -- (\x,\y) --  (\x-1.5158*0.5,\y + 1.75*.5 - .875*.5) -- (\x-1.5158,\y+.875); 
\draw[densely dashed, very thick, violet]  (\x+1.5158,\y+.875)  -- (\x-1.5158,\y+.875) node[pos=0.4,below] {$x_\gamma$}; 
\draw [mygreen] (\x,\y+1.75) -- (\x+1.5158,\y-.875) node[pos=0.8] {$x_c$};
	}
} 
\end{tikzpicture}% end gamma 
\quad\quad
\begin{tikzpicture}[scale = \mypolygonscale]
%\tikzstyle{every node} = [font = \small]
\foreach \x in {0}
{
    \foreach \y in {-8}
    {
			\fill (\x,\y+1.75) circle (.05);
			%\fill (\x,\y+1.75) node [above] {\tiny{$1$}};
			\fill(\x+1.5158,\y+.875) circle (.05);
			%\fill (\x+1.5158,\y+.875) node [right] {\tiny{$2 $}};
			\fill (\x+1.5158,\y-.875) circle (.05);
			%\fill (\x+1.5158,\y-.875) node [right] {\tiny{$3$}};
			\fill(\x,\y-1.75) circle (.05);
			%\fill (\x,\y-1.75) node [below] {\tiny{$4$}};
			\fill(\x-1.5158,\y-.875) circle (.05);
			%\fill (\x-1.5158,\y-.875) node [left] {\tiny{$5$}};
			\fill (\x-1.5158,\y+.875) circle (.05);
			%\fill (\x-1.5158,\y+.875) node [left] {\tiny{$6$}};
		
			\draw [] (\x,\y+1.75)--(\x+1.5158,\y+.875); %v1 to v2
			\draw [] (\x+1.5158,\y+.875) -- (\x+1.5158,\y-.875); %v3 to v2
			\draw [] (\x+1.5158,\y-.875)--(\x,\y-1.75); %v4 to v3
			\draw [] (\x,\y-1.75) -- (\x-1.5158,\y-.875); %v5 to v4
			\draw [] (\x-1.5158,\y-.875)--(\x-1.5158,\y+.875); %v6 to v5
			\draw [] (\x-1.5158,\y+.875) -- (\x,\y+1.75); %v1 to v6
		
\draw [red] (\x,\y+1.75) -- (\x-1.5158,\y-.875) node[pos=0.7] {$x_a$};
\draw[->,densely dotted, very thick] (\x,\y) -- (\x-1.5158*0.5,\y + 1.75*.5 - .875*.5); 
%\draw ((\x+\x-1.5158)*0.5,(\y+\y1.75 - .875)*.5) -- (\x,\y+1.75); 
\draw [blue] (\x,\y+1.75) -- (\x,\y-1.75) node[pos=0.65] {$x_b$};
\draw[->,densely dotted, very thick]  (\x+1.5158*0.5,\y + 1.75*.5 - .875*.5) -- (\x,\y); 
\draw [mygreen] (\x,\y+1.75) -- (\x+1.5158,\y-.875) node[pos=0.7] {$x_c$};
	}
}
\end{tikzpicture}
\quad\quad
Let $Q_\gamma=$
\begin{tikzpicture}[yscale=\myyscale, xscale=\myxscale, line join=bevel,
>=latex, 
%font = \small
]
\def\posetedgecolor{blue}
\node(c) at (0,0) {{$c$}}; %{$1$};
\node(b) at (0,1) {{$b$}}; %{$2$};
\node(a) at (0,2) {{$a$}}; %{$3$};

\draw[->] (c) -- (b) ;%node[\posetedgecolor,pos=0.5] {$0$}; 
\draw[->] (b) -- (a) ;%node[\posetedgecolor,pos=0.5] {$1$}; 
\end{tikzpicture}

Motivation: 
\begin{theorem}[\small Musiker, Schiffler, and Williams, 2011]
The order filters of $Q_\gamma$ are in bijection with the terms of the cluster variable expansion of $x_\gamma$ with respect to $x_a$, $x_b$, and $x_c$ 
(where the set of elements of each order filter corresponds to a term in the $F$-polynomial). 
\end{theorem}
}

\backupend

\end{document}