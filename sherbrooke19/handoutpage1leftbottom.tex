\documentclass{beamer}
\beamertemplatenavigationsymbolsempty


%\mode<presentation> {\usecolortheme{rose}
%}

\usepackage{xcolor}
\usepackage{graphicx}
\usepackage{wrapfig}

\usepackage[all]{xy} % to use xymatrix

%\usepackage{mathtools} % to put labels above leftrightarrow

%\usepackage{collectbox}
%
%\makeatletter
%\newcommand{\mybox}{%
%    \collectbox{%
%        \setlength{\fboxsep}{1pt}%
%        \fbox{\BOXCONTENT}%
%    }%
%}
%\makeatother
%
%
%\defbeamertemplate{footline}{author and frame number}{%
%  \usebeamercolor[fg]{frame number in head/foot}%
%  \usebeamerfont{frame number in head/foot}%
%  \hspace{1em}\insertshortauthor\hfill%
%%  \insertframenumber / 20
% % \insertframenumber\,/31\,\kern1em\vskip2pt%
%% \insertframenumber\,/\,\inserttotalframenumber
%}
%%\setbeamertemplate{footline}[author and frame number]{}

%%%%%%%%%%%%%
%%%%%%%% shortcuts %%%


% \definecolor{myblue}{cmyk}{1.00,0.56,0.00,0.34}
\definecolor{mygreen}{cmyk}{0.5,0,0.5,0.5}
% \definecolor{myred}{cmyk}{0.00,1.00,0.63,0.00}
% \definecolor{myyellow}{cmyk}{0.00,0.15,1.00,0.00}

\begin{document}



%
%
%\frame{
%\frametitle{Conway -- Coxeter friezes}
%\vspace{-2mm}
%\begin{definition}
%A (Conway -- Coxeter) \myemph{frieze} is an array of positive integers such that: \begin{enumerate}
%\itemsep-.2em
%\item it is bounded above and below by a row of $1$s
%\item
% every diamond $$
% \begin{array}{ccccccc}
% &b&\\[-1pt]
% a&&d\\[-1pt]
% &c&
%\end{array}$$ satisfies the 
%rule
%$ad-bc=1$.
%\end{enumerate}
%\end{definition}
%\vspace{-3mm}
%\begin{example} %[an integer frieze]
%\vspace {-8mm}
%\small
%\input{friezepentagonint}
%\end{example}
%\vspace{-2mm}
%Note: every frieze is completely determined by the 2nd row.
%}

% \newcommand{\pentagon}
% {
% \begin{scope}
% \coordinate (a) at (0,0);
%\foreach \x in {-144,-72,0,72,144} {
%   \begin{scope}[rotate=\x]
%    \node (\x) at (0,-1.25) [fill,circle,inner sep=0.5pt] {};
%   \end{scope}
%}
%\draw (-144) -- (-72) -- (0) -- (72) -- (144) -- (-144);
%\draw (0) node [below] { $v_1$};
%\draw (72) node [right] { $v_2$};
%\draw (144) node [right] { $v_3$};
%\draw (-144) node [left] { $v_4$};
%\draw (-72) node [left] { $v_5$};
%\draw (0) -- (144) (-144) -- (0);
%\end{scope}
%}
%
% \newcommand{\pentagonab}
% {
% \begin{scope}
% \coordinate (a) at (0,0);
%\foreach \x in {-144,-72,0,72,144} {
%   \begin{scope}[rotate=\x]
%    \node (\x) at (0,-1.25) [fill,circle,inner sep=0.5pt] {};
%   \end{scope}
%}
%\draw (-144) -- (-72) -- (0) -- (72) -- (144) -- (-144);
%\draw (0) node [below] { $v_1$};
%\draw (72) node [right] { $v_2$};
%\draw (144) node [right] { $v_3$};
%\draw (-144) node [left] { $v_4$};
%\draw (-72) node [left] { $v_5$};
%\draw (0) -- (144) node[pos=0.5,red]{\bf $a$} (-144) -- (0) node[pos=0.5,magenta!60]{\bf $b$};
%\end{scope}
%}
%
% \newcommand{\pentagonnum}
% {
% \begin{scope}
% \coordinate (a) at (0,0);
%\foreach \x in {-144,-72,0,72,144} {
%   \begin{scope}[rotate=\x]
%    \node (\x) at (0,-1.25) [fill,circle,inner sep=0.5pt] {};
%   \end{scope}
%}
%\draw (-144) -- (-72) -- (0) -- (72) -- (144) -- (-144);
%\draw (0) node [below] {\textcolor{blue}{\huge $\mathbf 3$}}; %v_1
%\draw (72) node [right] {\textcolor{red}{\huge $\mathbf 1$}}; % v2
%\draw (144) node [right] {\textcolor{green}{\huge $\mathbf 2$}}; % v3
%\draw (-144) node [left] {\textcolor{violet}{\huge $\mathbf 2$}}; % v4
%\draw (-72) node [left] {\textcolor{magenta!50}{\huge $\mathbf 1$}}; %v5
%\draw (0) -- (144) (-144) -- (0);
%\end{scope}
%}
%
%\newcommand{\pentagonvar}
% {
% \begin{scope}
% \coordinate (a) at (0,0);
%\foreach \x in {-144,-72,0,72,144} {
%   \begin{scope}[rotate=\x]
%    \node (\x) at (0,-1.25) [fill,circle,inner sep=0.5pt] {};
%   \end{scope}
%}
%\draw (-144) -- (-72) -- (0) -- (72) -- (144) -- (-144);
%\draw (0) node [below] {\textcolor{blue}{\large$\frac{1+x_2+x_1}{x_2\,x_1}$}}; %v_1
%\draw (72) node [right] {\textcolor{red}{\huge $x_2$}}; % v2
%\draw (144) node [right] {\textcolor{green}{\huge $2$}}; % v3
%\draw (-144) node [left] {\textcolor{violet}{\huge $2$}}; % v4
%\draw (-72) node [left] {\textcolor{magenta!50}{\huge $x_2$}}; %v5
%\draw (0) -- (144) (-144) -- (0);
%\end{scope}
%}
%




%
%
%
%
%\begin{frame}
%\frametitle{Cluster algebras (Fomin -- Zelevinsky, 2000)
%}
%\underline{Definition} A \myemph{cluster algebra} is a commutative ring with a distinguished set of generators, called \myemph{cluster variables}.\\
%
%\underline{Theorem} (Caldero -- Chapoton (2006):
%The cluster variables of a cluster algebra 
%from a triangulated polygon (type $A$)
% form a frieze.
%%(Proof: Caldero-Chapoton formula.)
%%\end{theorem}
%%Example: Type $A_2$
%\vspace{-6mm}
%% Well-known in type $A$ (polygon triangulation): the cluster variables form a frieze.
%
%\begin{equation*}
% \begin{array}{ccccccccccccccccccccc}
%\cdots&&1&& 1&&1&&1&&1&&1&&\cdots
% \\[4pt]
%&\textcolor{blue}{\frac{1+a+b}{a\,b}}&&
%\textcolor{red}{a}&&\textcolor{green}{\frac{1+b}{a}}&&
%\textcolor{violet}{\frac{1+a}{b}}&&\textcolor{magenta!50}{b}&&\textcolor{blue}{\frac{1+a+b}{a\,b}}
% \\[4pt]
%\cdots&&\textcolor{violet}{\frac{1+a}{b}}&&\textcolor{magenta!50}{b}&&\textcolor{blue}{\frac{1+a+b}{a\,b}}&&\textcolor{red}{a}&&\textcolor{green}{\frac{1+b}{a}}&&\textcolor{violet}{\frac{1+a}{b}}&&\cdots
% \\[4pt]
%&1&&1&&1&&1&&1&&1&&1
%\end{array}
%\end{equation*}
%\vspace {-4mm}
%(Example: type $A_2$)\\
%% \begin{figure}
%\vspace {-2mm}
%\begin{center}
% \begin{tikzpicture}[scale=0.7]
%\pentagonab
% \end{tikzpicture}
% \begin{tikzpicture}[scale=0.7]
% \pentagonnum
% \end{tikzpicture}
% \end{center}
%%   \begin{tikzpicture}[scale=0.8]
%% \pentagonvar
%%  \end{tikzpicture}
%%\end{figure}
%\vspace {-5mm}
% \begin{itemize}
%\item Remark: If the variables are specialized to $1$, we recover the Conway -- Coxeter integer frieze. 
%%If specialized to  nonzero numbers, we get a frieze with nonzero real numbers.
%\end{itemize}
%\end{frame}
%



\newcommand{\zehn}{\hspace{10pt}}
\newcommand{\fuenf}{\hspace{5pt}}
\newcommand{\fuenfm}{\hspace{-5pt}}



\begin{frame}
\frametitle{Positive integral friezes}
\noindent 
%\textbf{Positive integral friezes}
%\vskip -0.5cm
%\begin{small}
Setting $x_1$=$x_2$=$x_3$=$1$ produces 
a Conway -- Coxeter frieze pattern 
%\end{small}
%\begin{small}
\large
\[\xymatrix@R-15pt@C30pt{ 
1 &&
1 &&
1 &&
1 &&
\\
& 1  &&
3  &&
2  &&
1  &&
\\
1  &&
2  &&
5  &&
1  &&
 \\
&
1 &&
3 &&
2 &&
1 &&
\\
1 &&
1 &&
1 &&
1 &&
}
\] 
%\end{small}
\begin{itemize}
\item
The above frieze corresponds to the frieze vector $(1,1,1)$ relative to $Q=1 \rightarrow 2 \leftarrow 3$. 

\item
Given any type $\mathbb{A}_3$ quiver, there are 14 integer frieze vectors (whose values depend on the quiver). 
\end{itemize}

\end{frame}






\end{document}


